\section{Meilenstein 3 $-$ Spezifikation, Implementierung
  und Demo eines REST-API}

% Teilaufgabe 1 {{{
\subsection{Teilaufgabe 1: Festlegen von Aggregates}

\input{ms3/aggregates}

Wir sind der Meinung, dass sich die Datenobjekte Gericht,
Speise, Zubereitungsanleitung und Zutatenposition als ein
Aggregate mit Gericht als Aggregate Root zusammenfassen
lassen, da keine Referenzen auf innere Entities existieren
und ein fachlicher Zusammenhang besteht, da ein Gericht aus
Speisen besteht, Speisen eine Zubereitungsanleitung haben
und diese wiederum Zutatenpositionen, ergibt sich hier
ein enges fachliches Geflecht.

Eine m\"ogliche Invariante w\"are, wenn $Gericht.name$ eine
Kombination von den zugeh\"origen Speisen w\"are. Als
Beispiel hierf\"ur: $Gericht.name$: "`Schnitzel mit Pommes
und Salat"'. Daraus lassen sich die Speisen Schnitzel,
Pommes und Salat ableiten.
% }}}

% Teilaufgabe 2 {{{
\subsection{Teilaufgabe 2: Design des REST-API}

F\"ur unser REST-API verwenden wir folgenden Ausschnitt aus
unserem Klassendiagramm aus Meilenstein 1:

\begin{figure}[H]
\begin{center}
\begin{tikzpicture}

  \UMLClassAlterName
    {0}
    {0}
    {
      \textbf{Gericht}
        \nodepart{second}
      Name: String \\
      Details: String \\
      Preis: double
    }
    {gericht}
    {minimum width=6cm,text width=2.75cm, label=A/B}

  \UMLClassRelativeToAlterName
    {below=3 of gericht}
    {\textbf{Speise}\nodepart{second}Name: String}
    {speise}
    {minimum width=6cm,label={below:C}}

  \draw[umlaggreg-] (gericht.south) -- (speise.north);
  \node[below left=.25 of gericht.south] {*};
  \node[above left=.25 of speise.north] {*};
  \node[left] at ($(gericht)!.5!(speise)$) {Teil von};

\end{tikzpicture}
\end{center}
\caption{Ausschnitt Klassendiagramm f\"ur REST-API}
\end{figure}


\begin{tabu} to \linewidth {l|l|X|X|X}
% Headerzeile {{{
\hline
\rowcolor{codebordercolor}
Szen.-Nr. &URI &HTTP Verb &Request-Body &Ressource und
  Aktion \\
% }}}
% A1, A5, BC1 {{{
\hline
A1, A5, BC1 &/gerichte/\{gericht\_id\} &PUT, DELETE
  &\textbf{Nur bei Put:}
    \{
      name=\dots,
      details=\dots,
      preis=\dots
    \}
  &Neues Gericht anlegen, Gericht
  \"uberschreiben, Gericht l\"oschen. \\
% }}}
% A2, A4 {{{
\hline
A2, A4 &/gerichte?filter[preis]$>$\{wert\} &GET & &Alle
  Gerichte $a$ ausgeben, f\"ur die $a.preis > wert$ gilt.
  \\
% }}}
% A3 {{{
\hline
A3 &/gerichte/\{gericht\_id\}/preis &PUT &\{wert\} &Preis
  von Gericht $a$ mit $a.gericht\_id=\{gericht\_id\}$ auf
  $\{wert\}$ setzen. \\
% }}}
% A6 {{{
\hline
A6 &/gerichte/\{gericht\_id\}/id &GET & &Id von Gericht
  abfragen. 404 wird geworfen, falls Gericht nicht
  vorhanden. \\
% }}}
% BC2 {{{
\hline
BC2 &/speisen/\{speise\_id\} &PUT &\{name=\dots\} &Speise
  anlegen, \"uberschreiben. \\
% }}}
% BC3, BC6 {{{
\hline
BC3, BC6 &/gerichte/\{gericht\_id\}/speisen &PUT, DELETE
  &\{speise\_id=\dots\} &Speise einem Gericht hinzuf\"ugen
  oder l\"oschen.
  \\
% }}}
% BC4, BC7 {{{
\hline
BC4, BC7 &/gerichte &GET & &Alle Gerichte ausgeben. \\
% }}}
% BC5 {{{
\hline
BC5 &/speisen &GET & &Alle Speisen ausgeben. \\
% }}}
\hline
\end{tabu}
% }}}

\subsubsection*{Bemerkung}

Bei Szenario BC3 und BC6 haben wir uns entschieden, dass
die Beziehung zwsichen einer Instanz von Gericht und einer
Instanz von Speise \"uber /gerichte/\{gericht\_id\}/speisen
hinzugef\"ugt oder gel\"oscht weren kann. Man h\"atte dies
auch \"uber /speisen/\{speisen\_id\}/gerichte tun k\"onnen,
was wir jedoch f\"ur un\"ubersichtlicher und nicht so
naheliegend wie unsere Variante gehalten haben.

% Teilaufgabe 3 {{{
\subsection{Teilaufgabe 3: Implementierung in Spring Data
  JPA / Web MVC}

% }}}
