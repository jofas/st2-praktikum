\documentclass[12pt, twoside]{article}
\usepackage{amsmath}
\usepackage{amssymb}
\usepackage[colorlinks=true, linkcolor=black]{hyperref} % Links
\usepackage{makeidx} % Indexierung
\usepackage{siunitx}
\usepackage[ngerman]{babel} % deutsche Sonderzeichen
\usepackage[utf8]{inputenc}
\usepackage{geometry} % Dokumentendesign wie Seiten- oder Zeilenabstand bestimmen
\usepackage[toc,page]{appendix}

% Graphiken
\usepackage{tikz}
\usepackage{pgfplots}
\usepackage{pgfcore}
\usepackage{pgfopts}
\usepackage{pgf}
\usepackage{ifthen}
\usepackage{booktabs}

% Tabellen
\usepackage{tabu}
\usepackage{longtable}
\usepackage{colortbl} % Tabellen faerben
\usepackage{multirow}
\usepackage{diagbox} % Tabellenzelle diagonal splitten

\usepackage{xcolor} % Farben
\usepackage[framemethod=tikz]{mdframed} % Hintergrunderstellung
\usepackage{enumitem} % Enumerate mit Buchstaben nummerierbar machen
\usepackage{pdfpages}
\usepackage{listings} % Source-Code darstellen
\usepackage{eurosym} % Eurosymbol
\usepackage[square,numbers]{natbib}
\usepackage{here} % figure an richtiger Stelle positionieren
\usepackage{verbatim} % Blockkommentare mit \begin{comment}...\end{comment}
\usepackage{ulem} % \sout{} (durchgestrichener Text)

% BibLaTex
\bibliographystyle{acm}

% Aendern des Anhangnamens (Seite und Inhaltsverzeichnis)
\renewcommand\appendixtocname{Anhang}
\renewcommand\appendixpagename{Anhang}

% mdframed Style
\mdfdefinestyle{codebox}{
	linewidth=2.5pt,
	linecolor=codebordercolor,
	backgroundcolor=codecolor,
	shadow=true,
	shadowcolor=black!40!white,
	fontcolor=black,
	everyline=true,
}

% Seitenabstaende
\geometry{left=15mm,right=15mm,top=15mm,bottom=20mm}

% TikZ Bibliotheken
\usetikzlibrary{
    arrows,
    arrows.meta,
    decorations,
    backgrounds,
    positioning,
    fit,
    petri,
    shadows,
    datavisualization.formats.functions,
    calc,
    shapes.multipart
}

\pgfplotsset{width=7cm,compat=1.15}

\definecolor{codecolor}{HTML}{EEEEEE}
\definecolor{codebordercolor}{HTML}{CCCCCC}

% Standardeinstellungen fuer Source-Code
\lstset{
    language=C,
    breaklines=true,
    keepspaces=true,
    keywordstyle=\bfseries\color{green!70!black},
    basicstyle=\ttfamily\color{black},
    commentstyle=\itshape\color{purple},
    identifierstyle=\color{blue},
    stringstyle=\color{orange},
    showstringspaces=false,
    rulecolor=\color{black},
    tabsize=2,
    escapeinside={\%*}{*\%},
}

\input{libuml}
%\input{liberm}

\begin{document}

% Custom Titelseite
\begin{titlepage}
	\begin{center}
		{\Huge{\underline{\textbf{Softwaretechnik II $-$ Praktikum}}}} \\
	\vspace{3cm}
		{\Huge{\textbf{Subsystem 4 $-$ Zubereitung}}} \\
	\vspace{3cm}
		{\Huge{Eine Dokumentation von:}} \\
	\vspace{2cm}
		\huge{J. Fa{\ss}bender} \\
	\vspace{.5cm}
		\huge{J. Gobelet} \\
	\vspace{.5cm}
		\huge{L. Gobelet} \\
	\vspace{.5cm}
		\huge{E. G\"odel} \\
	\vspace{2cm}
    \Huge{Team 4.4}
	\end{center}
\end{titlepage}

\tableofcontents
\newpage
\listoffigures

\section{Meilenstein 1 $-$ Datenzugriffsschicht}

\subsection{Teilaufgabe 1: Ausschnitt aus Logischem DM mit Entities und Value Objects}

\subsubsection{Klassendiagramm}
\begin{center}
\begin{tikzpicture}

% Ebene 0 {{{
  \UMLClassAlterName
    {0}
    {0}
    {
      \textbf{Gericht}
        \nodepart{second}
      Name: String \\
      Details: String \\
      Preis: double
    }
    {gericht}
    {minimum width=6cm,text width=2.75cm}

  \UMLClassRelativeToAlterName
    {right=3 of gericht}
    {\textbf{Zutat}\nodepart{second}Name: String}
    {zutat}
    {minimum width=6cm}
% }}}

% Ebene -1 {{{
  \UMLClassRelativeToAlterName
    {below=3 of gericht}
    {\textbf{Speise}\nodepart{second}Name: String}
    {speise}
    {minimum width=6cm}
% }}}

% Ebene -2 {{{
  \UMLClassRelativeToAlterName
    {below=3 of speise}
    {
      \textbf{Zubereitungsanleitung}
        \nodepart{second}
      Anleitung: String
    }
    {za}
    {minimum width=6cm,fill=red!25}

  \UMLClassRelativeToAlterName
    {right=3 of za}
    {\textbf{Zutatenangabe}\nodepart{second}Menge: int}
    {zutatenangabe}
    {minimum width=6cm,fill=red!25}
% }}}

% Verbindungen {{{
  % gericht -- speise {{{
  \draw[umlaggreg-] (gericht.south) -- (speise.north);
  \node[below left=.25 of gericht.south] {*};
  \node[above left=.25 of speise.north] {*};
  \node[left] at ($(gericht)!.5!(speise)$) {Teil von};
  % }}}
  % speise -- za {{{
  \draw[umlcompo-] (speise.south) -- (za.north);
  \node[below left=.25 of speise.south] {1};
  \node[above left=.25 of za.north] {1};
  \node[left] at ($(speise)!.5!(za)$) {beschreibt};
  % }}}
  % zutatenangabe -- za {{{
  \draw[-umlcompo] (zutatenangabe.west) -- (za.east);
  \node[above left=.25 of zutatenangabe.west] {1..*};
  \node[above right=.25 of za.east] {1};
  \node[below] at ($(zutatenangabe)!.5!(za)$) {ben\"otigt};
  % }}}
  % zutatenangabe -- zutat {{{
  \draw (zutatenangabe.north) -- (zutat.south);
  \node[above left=.25 of zutatenangabe.north] {*};
  \node[below left=.25 of zutat.south] {1};
  \node[left] at ($(zutatenangabe)!.5!(zutat)$) {ben\"otigt};
  % }}}
% }}}
\end{tikzpicture}
\end{center}


\subsubsection{Fachliches Glossar}
\begin{tabu} to \linewidth {X|X|X}
% Headerzeile {{{
\hline
\rowcolor{codebordercolor}
Gesch\"aftsobjekt &Attribut &Erkl\"arung \\
% }}}
% Gericht {{{
\hline
Gericht & &Vom Restaurant angebotenes Mahl. \\
  % Name {{{
  \hline
  &Name &Gerichtsbezeichnung. \\
  % }}}
  % Details {{{
  \hline
  &Details &Wird dem Gast angezeigt. Enth\"alt n\"ahere Angaben zu den Zutaten. \\
  % }}}
  % Preis {{{
  \hline
  &Preis &Geldbetrag der f\"ur das Gericht zu bezahlen ist. \\
  % }}}
% }}}
% Speise {{{
\hline
Speise & &Teil eines Gerichts. Beispielsweise w\"are eine Salatbeilage als Speise zu verstehen. \\
  % Name {{{
  \hline
  &Name &Bezeichnung der Speise. \\
  % }}}
% }}}
% Zubereitungsanleitung {{{
\hline
Zubereitungsanleitung & &Leitfaden zur Zubereitung einer Speise. \\
  % Anleitung {{{
  \hline
  &Anleitung &Erkl\"arender Text, der beschreibt, wie eine Speise zuzubereiten ist. \\
  % }}}
% }}}
% Zutat {{{
\hline
Zutat & &Ben\"otigt f\"ur die Zubereitung einer Speise. \\
  % Name {{{
  \hline
  &Name &Bezeichnung der Zutat. \\
  % }}}
% }}}
% Zutatenangabe {{{
\hline
Zutatenangabe & &Zuordnung zwischen Zutat und Zubereitungsanleitung.
                 Gibt die Menge einer Zutat an, die f\"ur die Zubereitung notwendig ist. \\
  % Menge {{{
  \hline
  &Menge &Die ben\"otigte Menge. \\
  % }}}
% }}}
\hline
\end{tabu}


\subsubsection{Erweiterungen der Aufgabenstellung}

Da es in unserem Logischen Datenmodell keine 1:1-Beziehung gab,
haben wir eine zus\"atzliche redundante Entit\"at eingebaut.

Hierbei handelt es sich um die Entit\"at Speise. Diese Entit\"at
h\"atte genauso gut einfach Teil der Zubereitungsanleitung sein
k\"onnen und ist nur in unser Modell aufgenommen worden, damit
wir die f\"ur die Aufgabenstellung ben\"otigte 1:1-Beziehung in
unserem Diagramm haben.

\section{Meilenstein 2 $-$ Komponentenschnitt}
\label{ms2}

% Teilaufgabe 1 {{{
\subsection{Teilaufgabe 1: Vorbereitung des Komponentenschnitts}

% Liste der Geschaeftsobjekte {{{
\subsubsection{Liste der Gesch\"aftsobjekte}

\begin{itemize}

  \item Arbeitsplatz

  \item Bestellung

  \item Gericht

  \item Sitzplatz

  \item Speisekarte

  \item Zubereitungsanleitung

  \item Zutat

  \item Zutatenposition

\end{itemize}
% }}}

% Liste der Use Cases {{{
\subsubsection{Liste der Use Cases}

\begin{itemize}

  \item Am Arbeitsplatz an-/abmelden

  \item Gericht bestellen

  \item Gericht zubereiten

\end{itemize}
% }}}

% Liste der Umsysteme {{{
\subsubsection{Liste der Umsysteme}

\begin{tabu} to \linewidth {X|X|X}
% Headerzeile {{{
\hline
\rowcolor{codebordercolor}
Umsystem &Was geschieht zwischen Umsystem und unserem Subsystem?
  &Schnittstelle angeboten oder aufgerufen \\
% }}}
% Rezeptverwaltung {{{
\hline
Rezeptverwaltung &Rezeptverwaltung verwaltet die Gesch\"aftsobjekte Gericht,
  Zubereitungsanleitung und Speisekarte. Der Gast fragt \"uber das ihm zur
  Verf\"ugung gestellte Frontend die Speisekarte und die Gerichte ab, w\"ahrend
  der Koch an seinem Terminal die Zubereitungsanleitung und die hiermit verbundenen
  Zutatenpositionen, angezeigt bekommt.

  &Aufruf einer Schnittstelle zur Rezeptverwaltung \\
% }}}
% Lagerverwaltung {{{
\hline
Lagerverwaltung &Abfrage zum Zutatenbestand &Aufruf einer Schnittstelle zur
  Lagerverwaltung \\
% }}}
% Lagerverwaltung {{{
\hline
Lagerverwaltung &Angabe zur Zutantenentnahme (kann auch \"uber die gleiche
  Schnittstelle, die im obrigen Tabelleneintrag spezifiziert ist, realisiert
  werden)

  &Aufruf einer Schnittstelle zur Lagerverwaltung \\
% }}}
% Buchhaltung {{{
\hline
Buchhaltung &Abfrage der Bestellungen &Schnittstelle wird Buchhaltung zur verf\"ugung
  gestellt \\
% }}}
\hline
\end{tabu}

% Erlaeuterung {{{
\textbf{Erl\"auterung} \\

Wir legen redundant zur Lagerverwaltung unsere eingene Verwaltung
mit Angaben zum Zutatenbestand an, um auch bei Nichterreichbarkeit
der Lagerverwaltung funktionsf\"ahig zu bleiben, da unser Subsystem
essentiell f\"ur den Umsatz verantwortlich ist und ein Ausfall, das
hei{\ss}t in diesem Fall der Zustand, dass eine Zutat nicht mehr in
ben\"otigter Menge im Lager zur Verf\"ugung steht, nicht auf Grund
technischer Probleme eintreten sollte.

Allerdings stellen wir keinen Anspruch auf absolute Richtigkeit unserer
Zutantenbestandsverwaltung, da wir nur die Ereignisse unseres Subsystems,
das hei{\ss}t in diesem Fall die Entnahme einer Zutat zur Zubereitung,
protokollieren und die restlichen Angaben aus der Lagerverwaltung stammen.

Ist diese nun nicht erreichbar, verwendet unsere Zutatenbestandsverwaltung
mitunter veraltete Daten, was wir nicht mit einbeziehen.

Der Lagerverwaltung wird die Entnahme von unserem Subsystem aus mitgeteilt.

F\"ur den kompletten Synchronisationsprozess zwischen den beiden Systemen
stellt uns die Lagerverwaltung zwei Schnittstellen (oder eine, die beide
Aufgaben - Entnahme mitteilen und Zutatenbestand abfragen - zusammenfasst)
zur Verf\"ugung.

Zus\"atzlich haben wir eine Schnittstelle f\"ur die Buchhaltung angelegt.
Diese ist zwar kein explizites Subsystem, wird aber, unserer Meinung nach,
im Betriebsumfeld h\"ochstwahrscheinlich als eigenes Subsystem existieren
und unsere Schnittstelle zu den Bestellungen (im Endeffekt der Unternehmens\-
umsatz aus dem Hauptgesch\"aft) nutzen wollen.
% }}}

% }}}

% }}}

% Teilaufgabe 2 {{{
\subsection{Teilaufgabe 2: Ermittlung der verschiedenen Komponenten-Typen}

% Schritt 1 {{{
\subsubsection{Schritt 1: Gesch\"aftsobjekte in zusammenh\"angende Gruppen einteilen}
\label{gos}

\begin{tabu} to \linewidth {X|X|X}
% Headerzeile {{{
\hline
\rowcolor{codebordercolor}
Datenkomponente &Zugeordnete Gesch\"aftsobjekte &Erkl\"arung \\
% }}}
% Bestellungskomponente {{{
\hline
Bestelldaten &Bestellung &Die einzigen Daten die in diesem Subsystem tas\"achlich generiert
  werden. Da die Bestellungen sehr wichtig f\"ur das Hauptgesch\"aft der Firma ist, es das einzige
  Datenobjekt mit Implementierung eines Create-Interfaces (Factory) ist und auch sonst nicht in unsere
  sonstigen Datenkomponenten passt, wird die Bestellung, unserer Meinung nach, in einer eigenen
  Komponente implementiert.  \\
% }}}
% Standortkomponente {{{
\hline
Standortdaten &Arbeitsplatz, Sitzplatz &Diese Daten \"andern sich \"au{\ss}erst selten (und auch nicht
  in unserem Subsystem) und umfassen im Vergleich zu anderen Komponenten wenig Datens\"atze und k\"onnen
  deshalb, unserer Meinung nach, gut zusammengefasst werden.\\
% }}}
% Gerichtskomponente {{{
\hline
Gerichtsdaten &Gericht, Speisekarte, Zubereitungsanleitung, Zutat, Zutatenposition
  &Stammdaten die f\"ur unseren Prozess der Zubereitung essentiell sind. Diese Daten
  stammen nicht aus unserem Subsystem, sondern sind \"uber Schnittstellen abrufbar,
  sowohl von der Lagerverwaltung (Zutat), als auch von der Rezepteverwaltung (Gericht,
  Speisekarte, Zubereitungsanleitung, Zutatenposition). Unsere Datenkomponente greift
  \"uber Adapterkomponenten auf diese Schnittstellen zu.\\
% }}}
\hline
\end{tabu}

% }}}

% Schritt 2 {{{
\subsubsection{Schritt 2: Use Cases auf Daten/Logik analysieren}

\begin{tabu} to \linewidth {X|X|X}
% Headerzeile {{{
\hline
\rowcolor{codebordercolor}
Daten-/Logikkomponente &Zugeordnete(r) Use Case(s) &Erkl\"arung \\
% }}}
% Bestellabwicklung {{{
\hline
Bestellabwicklung (Logik) &Am Arbeitsplatz an-/abmelden, Gericht bestellen, Gericht zubereiten
  &Unser "`Backend"', was ab der Bestellungsaufgabe den Zubereitungsprozess steuert. Die
  Komponente umfasst die Vergabewarteschlange mit den besetzten und freien Arbeitspl\"atzen
  und \"ubernimmt die Zuweisung, sobald eine Bestellung von einem Clienten eingeht. Sobald ein
  Gericht fertig zubereitet ist und der Koch dies seinem Terminal mitteilt, \"ubernimmt diese
  Komponente auch die Anzeige der Ordernummer (im Gast-UI). Da dies alles vom Umfang her eher
  kleinere Aufgaben sind, haben wir uns dazu entschieden, diese Aufgaben in einer Komponente
  zusammenzufassen.\\
% }}}
\hline
\end{tabu}

% }}}

% Schritt 3 {{{
\subsubsection{Schritt 3: Use Cases auf Nutzer-Interaktion analysieren}

\begin{tabu} to \linewidth {X|X|X|X}
% Headerzeile {{{
\hline
\rowcolor{codebordercolor}
Dialogkomponente &Zugeordnete(r) Use Case(s) &Eigene Fassadenkomponente sinnvoll?
  &Erkl\"arung \\
% }}}
% Zubereitungs-UI {{{
\hline
Zubereitungs-UI &Gericht zubereiten &Ja &Fassadenkomponente zur
  Orchestrierung der Gerichtszubereitung. \\
% }}}
% An-/Abmeldungs-UI {{{
\hline
An-/Abmeldungs-UI &Am Arbeitsplatz an-/abmelden &Ja &Fassadenkomponente f\"ur den Zugriff auf
  Datenkomponente "`Standortdaten"' (Read- und Updateoperationen auf den Arbeitsplatz) und
  um das "`Strict Layering"' einzuhalten.\\
% }}}
% Gast-UI {{{
\hline
Gast-UI &Gericht bestellen &Ja &Fassadenkomponente zur Orchestrierung des Bestellvorgangs. \\
% }}}
\hline
\end{tabu}

% }}}

% Schritt 4 {{{
\subsubsection{Schritt 4: Angebot von externen Schnittstellen}

\begin{tabu} to \linewidth {X|X|X}
% Headerzeile {{{
\hline
\rowcolor{codebordercolor}
Umsystem/Schnittstelle &Eigene Fassadenkomponente sinnvoll? &Erkl\"arung \\
% }}}
% Buchhaltung {{{
\hline
Buchhaltung &Ja &Da die Buchhaltung lesenden Zugriff auf
  usere Bestellungen haben soll, ist es notwendig eine spezialisierte Komponente
  hierf\"ur anzulegen und nicht, wie intern in unserem Subsystem, den Zugriff
  \"uber die Bestelldatenkomponente zu regeln.\\
% }}}
% Lagerverwaltung {{{
\hline
Lagerverwaltung &Nein &Zugriff erfolgt nur aus der Gerichtsdatenkomponente \"uber die
  Adapterkomponente der Lagerverwaltung, weshalb, unserer Meinung nach, keine Fassaden\-
  komponente notwendig ist. \\
% }}}
% Rezeptverwaltung {{{
\hline
Rezeptverwaltung &Nein &Zugriff erfolgt nur aus der Gerichtsdatenkomponente \"uber die
  Adapterkomponente der Rezepteverwaltung, weshalb, unserer Meinung nach, keine Fassaden\-
  komponente notwendig ist. \\
% }}}
\hline
\end{tabu}

% }}}

% Schritt 5 {{{
\subsubsection{Schritt 5: Aufruf von externen Schnittstellen/Umsystemen}

\begin{tabu} to \linewidth {X|X|X}
% Headerzeile {{{
\hline
\rowcolor{codebordercolor}
Umsystem/Schnittstelle &Adapterkomponente sinnvoll? &Erkl\"arung \\
% }}}
% Buchhaltung {{{
\hline
Buchhaltung &Nein &Bereits spezialisierte Fassadenkomponente vorhanden. \\
% }}}
% Lagerverwaltung {{{
\hline
Lagerverwaltung &Ja &Adapterkomponente f\"ur unsere Gerichtsdatenkomponente,
  die die Lese- und Schreibvorg\"ange zur Verf\"ugung stellt und gleichzeitig
  bei Ausf\"allen als "`Anti-Corruption-Layer"' fungiert. \\
% }}}
% Rezeptverwaltung {{{
\hline
Rezeptverwaltung &Ja &Adapterkomponente f\"ur unsere Gerichtsdatenkomponente,
  die die Lesevorg\"ange zur Verf\"ugung stellt und gleichzeitig bei Ausf\"allen
  als "`Anti-Corruption-Layer"' fungiert. \\
% }}}
\hline
\end{tabu}

% }}}

% }}}

% Teilaufgabe 3 {{{
\subsection{Teilaufgabe 3: Komponentendiagramm}
\label{ms2_3}
\begin{figure}[H]
\begin{center}
\begin{tikzpicture}
  % Column 1 {{{

  % Row 1 {{{
  \UMLSComponent
    {0}
    {0}
    {Gast-UI}
    {fill=green!25,text width=3cm}
  % }}}

  % Row 2 {{{
  \UMLSComponentRelativeToAlterName
    {below=3 of Gast-UI}
    {Bestellvorgangs\-orchestrierung}
    {bo}
    {fill=purple,text width=3cm}
  % }}}

  % Row 3 {{{
  \UMLSComponentRelativeTo
    {below=3 of bo}
    {Standortdaten}
    {fill=blue!25,text width=3cm}
  % }}}

  % Row 4 {{{
  \UMLSComponentRelativeToAlterName
    {below=3 of Standortdaten}
    {Buchhaltungs-API}
    {ba}
    {fill=purple,text width=3cm}
  % }}}

  % Row 5 {{{
  \UMLSComponentRelativeToAlterName
    {below=3 of ba}
    {$<<$Buchhal\-tung$>>$}
    {bh}
    {fill=black!50,text width=3cm}
  % }}}

  % }}}

  % Column 2 {{{

  % Row 1 {{{
  \UMLSComponentRelativeTo
    {right=3 of Gast-UI}
    {Backend}
    {fill=orange,text width=3cm}
  % }}}

  % Row 3 {{{
  \UMLSComponentRelativeTo
    {right=3 of Standortdaten}
    {Bestelldaten}
    {fill=blue!25,text width=3cm}
  % }}}

  % Row 4 {{{
  \UMLSComponentRelativeTo
    {below=3 of Bestelldaten}
    {Rezeptadapter}
    {fill=black!25,text width=3cm}
  % }}}

  % Row 5 {{{
  \UMLSComponentRelativeToAlterName
    {right=3 of bh}
    {$<<$Rezeptver\-waltung$>>$}
    {rvw}
    {fill=black!50,text width=3cm}
  % }}}

  % }}}

  % Column 3 {{{

  % Row 1 {{{
  \UMLSComponentRelativeTo
    {right = 3 of Backend}
    {Koch-UI}
    {fill=green!25, text width=3cm}
  % }}}

  % Row 2 {{{
  \UMLSComponentRelativeToAlterName
    {below=3 of Koch-UI}
    {Zubereitungs\-orchestrierung}
    {zo}
    {fill=purple,text width=3cm}
  % }}}

  % Row 3 {{{
  \UMLSComponentRelativeTo
    {below=3 of zo}
    {Gerichtsdaten}
    {fill=blue!25,text width=3cm}
  % }}}

  % Row 4 {{{
  \UMLSComponentRelativeToAlterName
    {below=3 of Gerichtsdaten}
    {Lagerverwal\-tungsadapter}
    {lvwa}
    {fill=black!25,text width=3cm}
  % }}}

  % Row 5 {{{
  \UMLSComponentRelativeToAlterName
    {below=3 of lvwa}
    {$<<$Lagerver\-waltung$>>$}
    {lvw}
    {fill=black!50,text width=3cm}
  % }}}

  % }}}

  % Connections {{{

    % Gast-UI -> bo {{{
    \draw [-{>[sep=3.5]>}] (Gast-UI.south) -- (bo.north);

    \UMLComponentPort
      {Gast-UI.south}
      {fill=white}

    \UMLComponentPort
      {bo.north}
      {fill=white}
    % }}}

    % bo -> Backend {{{
    \draw [-{>[sep=3.5]>}] (bo.east) -| (Backend.230);

    \UMLComponentPort
      {bo.east}
      {fill=white}

    \UMLComponentPort
      {Backend.230}
      {fill=white}
    % }}}

    % Backend -> zo {{{
    \draw [-{>[sep=3.5]>}] (Backend.310) |- (zo.west);

    \UMLComponentPort
      {Backend.310}
      {fill=white}

    \UMLComponentPort
      {zo.west}
      {fill=white}
    % }}}

    % Koch-UI -> zo {{{
    \draw [-{>[sep=3.5]>}] (Koch-UI.south) -- (zo.north);

    \UMLComponentPort
      {Koch-UI.south}
      {fill=white}

    \UMLComponentPort
      {zo.north}
      {fill=white}
    % }}}

    % Koch-UI -> Backend {{{
    \draw [-{>[sep=3.5]>}] (Koch-UI.west) -- (Backend.east);

    \UMLComponentPort
      {Koch-UI.west}
      {fill=white}

    \UMLComponentPort
      {Backend.east}
      {fill=white}
    % }}}

    % bo -> Bestelldaten {{{
    \draw [-{>[sep=3.5]>}]
      (bo.south)
      |-
      ($(bo)!.6!(Bestelldaten)$)
      -|
      (Bestelldaten.158);

    \UMLComponentPort
      {bo.south}
      {fill=white}

    \UMLComponentPort
      {Bestelldaten.158}
      {fill=white}
    % }}}

    % bo -> Standortdaten {{{
    \draw [-{>[sep=3.5]>}] (bo.210) -- (Standortdaten.158);

    \UMLComponentPort
      {bo.210}
      {fill=white}

    \UMLComponentPort
      {Standortdaten.158}
      {fill=white}
    % }}}

    % bo -> Gerichtsdaten {{{
    \draw [-{>[sep=3.5]>}]
      (bo.330)
      |-
      ($(bo)!.4!(Gerichtsdaten)$)
      -|
      (Gerichtsdaten.158);

    \UMLComponentPort
      {bo.330}
      {fill=white}

    \UMLComponentPort
      {Gerichtsdaten.158}
      {fill=white}
    % }}}

    % zo -> Gerichtsdaten {{{
    \draw [-{>[sep=3.5]>}] (zo.330) -- (Gerichtsdaten.22);

    \UMLComponentPort
      {zo.330}
      {fill=white}

    \UMLComponentPort
      {Gerichtsdaten.22}
      {fill=white}
    % }}}

    % zo -> Bestelldaten {{{
    \draw [-{>[sep=3.5]>}]
      (zo.south)
      |-
      ($(zo)!.6!(Bestelldaten)$)
      -|
      (Bestelldaten.22);

    \UMLComponentPort
      {zo.south}
      {fill=white}

    \UMLComponentPort
      {Bestelldaten.22}
      {fill=white}
    % }}}

    % ba -> Bestelldaten {{{
    \draw [-{>[sep=3.5]>}]
      (ba.north)
      |-
      ($(ba)!.7!(Bestelldaten)$)
      -|
      (Bestelldaten.south);

    \UMLComponentPort
      {ba.north}
      {fill=white}

    \UMLComponentPort
      {Bestelldaten.south}
      {fill=white}
    % }}}

    % Gerichtsdaten -> Rezeptadapter {{{
    \draw [-{>[sep=3.5]>}]
      (Gerichtsdaten.230)
      |-
      ($(Gerichtsdaten)!.5!(Rezeptadapter)$)
      -|
      (Rezeptadapter.north);

    \UMLComponentPort
      {Gerichtsdaten.230}
      {fill=white}

    \UMLComponentPort
      {Rezeptadapter.north}
      {fill=white}
    % }}}

    % Gerichtsdaten -> lvwa {{{
    \draw [-{>[sep=3.5]>}] (Gerichtsdaten.310) -- (lvwa.60);

    \UMLComponentPort
      {Gerichtsdaten.310}
      {fill=white}

    \UMLComponentPort
      {lvwa.60}
      {fill=white}
    % }}}

    % ba -> bh {{{
    \draw [-umlportprovider] (ba)--($(ba)!.5!(bh)$);
    \draw [-umlportcaller]   (bh)--($(bh)!.5!(ba)$);

    \UMLComponentPort
      {ba.south}
      {fill=white}

    \UMLComponentPort
      {bh.north}
      {fill=white}
    % }}}

    % Rezeptadapter -> rvw {{{
    \draw [-umlportcaller]   (Rezeptadapter)--($(Rezeptadapter)!.5!(rvw)$);
    \draw [-umlportprovider] (rvw)--($(rvw)!.5!(Rezeptadapter)$);

    \UMLComponentPort
      {Rezeptadapter.south}
      {fill=white}

    \UMLComponentPort
      {rvw.north}
      {fill=white}
    % }}}

    % lvwa -> lvw {{{
    \draw [-umlportcaller]   (lvwa)--($(lvwa)!.5!(lvw)$);
    \draw [-umlportprovider] (lvw)--($(lvw)!.5!(lvwa)$);

    \UMLComponentPort
      {lvwa.south}
      {fill=white}

    \UMLComponentPort
      {lvw.north}
      {fill=white}
    % }}}

  % }}}

\end{tikzpicture}
\caption{Komponentendiagramm}
\end{center}
\end{figure}

\newpage

% Legende {{{
\begin{itemize}[label={}]
  \item \tikz[baseline={-3pt}]{
    \node[fill=green!25,green!25,inner ysep=3pt]{X};
  }: Dialogkomponente

  \item \tikz[baseline={-3pt}]{
    \node[fill=purple,purple,inner ysep=3pt]{X};
  }: Fassadenkomponente

  \item \tikz[baseline={-3pt}]{
    \node[fill=blue!25,blue!25,inner ysep=3pt]{X};
  }: Datenkomponente

  \item \tikz[baseline={-3pt}]{
    \node[fill=orange,orange,inner ysep=3pt]{X};
  }: Logikkomponente

  \item \tikz[baseline={-3pt}]{
    \node[fill=black!25,black!25,inner ysep=3pt]{X};
  }: Adapterkomponente

  \item \tikz[baseline={-3pt}]{
    \node[fill=black!50,black!50,inner ysep=3pt]{X};
  }: Umsystem
\end{itemize}
% }}}

% }}}


\section{Meilenstein 3 $-$ Spezifikation, Implementierung
  und Demo eines REST-API}

% Teilaufgabe 1 {{{
\subsection{Teilaufgabe 1: Festlegen von Aggregates}%
\label{ms3_aggregates}

\input{ms3/aggregates}

Wir sind der Meinung, dass sich die Datenobjekte Gericht,
Speise, Zubereitungsanleitung, Zutatenposition und Zutat
als ein Aggregate mit Gericht als Aggregate Root
zusammenfassen lassen, da keine Referenzen auf innere
Entities existieren und ein fachlicher Zusammenhang
besteht, da ein Gericht aus Speisen besteht, Speisen eine
Zubereitungsanleitung haben und diese wiederum
Zutatenpositionen, die auf Zutaten verweisen, ergibt sich
hier ein enges fachliches Geflecht. Au{\ss}erdem ist es so,
dass wir in Meilenstein 2 alle diese Objekte in der
Datenkomponente Gerichtsdaten (vgl. \ref{gos})
zusammengefasst haben, weshalb wir uns \"uberlegt haben,
dass das Aggregate durchaus deckungsgleich sein k\"onnte.

Eine m\"ogliche Invariante w\"are, wenn $Gericht.name$ eine
Kombination von den zugeh\"origen Speisen w\"are. Als
Beispiel hierf\"ur: $Gericht.name$: "`Schnitzel mit Pommes
und Salat"'. Daraus lassen sich die Speisen Schnitzel,
Pommes und Salat ableiten.
% }}}

% Teilaufgabe 2 {{{
\subsection{Teilaufgabe 2: Design des REST-API}

F\"ur unser REST-API verwenden wir folgenden Ausschnitt aus
unserem Klassendiagramm aus Meilenstein 1:

\begin{figure}[H]
\begin{center}
\begin{tikzpicture}

  \UMLClassAlterName
    {0}
    {0}
    {
      \textbf{Gericht}
        \nodepart{second}
      Name: String \\
      Details: String \\
      Preis: double
    }
    {gericht}
    {minimum width=6cm,text width=2.75cm, label=A/B}

  \UMLClassRelativeToAlterName
    {below=3 of gericht}
    {\textbf{Speise}\nodepart{second}Name: String}
    {speise}
    {minimum width=6cm,label={below:C}}

  \draw[umlaggreg-] (gericht.south) -- (speise.north);
  \node[below left=.25 of gericht.south] {*};
  \node[above left=.25 of speise.north] {*};
  \node[left] at ($(gericht)!.5!(speise)$) {Teil von};

\end{tikzpicture}
\end{center}
\caption{Ausschnitt Klassendiagramm f\"ur REST-API}
\end{figure}


\subsubsection*{Erl\"auterung}

Diese Schnittstelle w\"urde so in unserem Subsystem nicht
implementiert, da die verwendeten Gesch\"aftsobjekte nicht
in unser Subsystem geh\"oren und wir sie deshalb selbst
\"uber Schnittstellen aus anderen Subsystemen beziehen.
Wir bieten nur eine Schnittstelle f\"ur das
Gesch\"aftsobjekt Bestellung f\"ur die Buchhaltung an (vgl.
\ref{ms2_3}) und mit nur einem Gesch\"aftsobjekt l\"asst
sich das angegebene Szenario nicht durchf\"uhren, weshalb
wir den obrigen Ausschnitt verwenden.


\begin{tabu} to \linewidth {p{1.8cm}|X|p{2.5cm}|p{3cm}|X}
% Headerzeile {{{
\hline
\rowcolor{codebordercolor}
Szen.-Nr. &URI &HTTP Verb &Request-Body &Ressource und
  Aktion \\
% }}}
% A1, BC1, BC4, BC7 {{{
\hline
A1, BC1, BC4, BC7 &/gerichte &POST, GET
  &\textbf{Nur bei POST:}
    \{
      name=\dots,
      details=\dots,
      preis=\dots
    \}
  &Neues Gericht anlegen, alle Gerichte ausgeben. \\
% }}}
% A2, A4 {{{
\hline
A2, A4 &/gerichte?search= preis$>$\{wert\} &GET & &Alle
  Gerichte $a$ ausgeben, f\"ur die $a.preis > wert$ gilt.
  \\
% }}}
% A3 {{{
\hline
A3 &/gerichte/\{gericht\_id\}/ preis &PUT &\{wert\} &Preis
  von Gericht $a$ mit $a.gericht\_id=\{gericht\_id\}$ auf
  $\{wert\}$ setzen. \\
% }}}
% A6 {{{
\hline
A6 &/gerichte/\{gericht\_id\} &GET & &Ein bestimmtes
  Gericht \"uber die Id
  abfragen. 404 wird geworfen, falls Gericht nicht
  vorhanden. \\
% }}}
% BC2, BC5 {{{
\hline
BC2 &/speisen &POST, GET
  &\textbf{Nur bei POST:}
  \{name=\dots\}
  &Neue Speise anlegen. Alle Speisen ausgeben. \\
% }}}
% BC3, BC6 {{{
\hline
BC3, BC6 &/gerichte/\{gericht\_id\}/ speisen/\{speise\_id\}
  &PUT, DELETE & &Speise einem Gericht hinzuf\"ugen oder
  l\"oschen. \\
% }}}
\end{tabu}

\subsubsection*{Erl\"auterung}

Bei Szenario BC3 und BC6 haben wir uns entschieden, dass
die Beziehung zwsichen einer Instanz von Gericht und einer
Instanz von Speise \"uber /gerichte/\{gericht\_id\}/speisen
hinzugef\"ugt oder gel\"oscht weren kann. Man h\"atte dies
auch \"uber /speisen/\{speisen\_id\}/gerichte tun k\"onnen,
was wir jedoch f\"ur un\"ubersichtlicher und nicht so
naheliegend wie unsere Variante gehalten haben.

% }}}

\newpage

% Teilaufgabe 3 {{{
\subsection{Teilaufgabe 3: Implementierung in Spring Data
  JPA / Web MVC}

% Code-Listing {{{
\subsubsection{Code-Listing}

% GerichtRESTController {{{
\begin{mdframed}[style=codebox]
\textbf{GerichtRESTController}
\lstinputlisting[
  language=Java,
  firstline=34,
  lastline=152
]{%
  ../code/ms3RESTSpringDemo/src/main/java/%
  ms3restspringdemo/services/GerichtRestController%
  .java
}
\end{mdframed}
% }}}

% SpeiseRESTController {{{
\begin{mdframed}[style=codebox]
\textbf{SpeiseRESTController}
\lstinputlisting[
  language=Java,
  firstline=28,
  lastline=47
]
{%
  ../code/ms3RESTSpringDemo/src/main/java/%
  ms3restspringdemo/services/SpeiseRestController%
  .java
}
\end{mdframed}
% }}}

% }}}

% Custom JSON Serializer {{{
\subsubsection{Custom JSON Serializer}

%Vermeidung von Endlosschleifen bei der JSON-Ausgabe

Um das Problem der Endlosserialisierung bei der m:n-
Beziehung, die bei uns zwischen Gericht und Speise
vorliegt, zu l\"osen, wurde vorgeschlagen die Annotation
"`@JsonIdentityInfo"' zu benutzen, mit der man die
jeweiligen Klassen annotieren muss.

Die Serialisierung funktioniert dann so, dass beim ersten
Vorkommen einer Entit\"at diese ganz serialisiert wird,
beim zweiten Vorkommen allerdings nur über die ID
referenziert wird.

% {{{
\begin{mdframed}[style=codebox]
\textbf{Beispielausgabe der Gerichte}
\begin{lstlisting}
[
    {
        "id": 2,
        "name": "Schnitzel mit Pommes",
        "details": "Der grandiose Klassiker!",
        "preis": 12.5,
        "speisen": [
            {
                "id": 2,
                "name": "Schnitzel",
                "gerichte": [
                    2
                ],
            },
            {
                "id": 1,
                "name": "Pommes",
                "gerichte": [
                    2,
                    {
                        "id": 3,
                        "name": "Grosse Portion Pommes",
                        "details": "Lecker fettig!",
                        "preis": 6,
                        "speisen": [
                            1
                        ]
                    }
                ],
            }
        ]
    },
    3
]
\end{lstlisting}
\end{mdframed}
% }}}

Aus unserer Sicht ist die schlecht lesbare Ausgabe ein
Nachteil.

Ein Beispiel f\"ur die schlechte Lesbarkeit w\"are unserer
Meinung nach die "`3"' ganz unten (vorletzte Zeile, s.o.),
welche f\"ur das Gericht mit der ID 3 (Grosse Portion
Pommes) steht, welches bereits zuvor aufgef\"uhrt wurde.

Eine andere L\"osung, die wir gefunden haben ist ein
Custom-Serialisierer.

Hierbei k\"onnen wir dann selbst bestimmen, wie die
Serialisierung funktionieren soll.

Hier haben wir uns dafür entschieden nur einen
Serialisierer für die Gerichte, auf den die Speisen
verweisen, zu programmieren, der anstatt die Gerichte in
die "`Tiefe"' zu serialisieren, nur die IDs der Gerichte
auflistet. Damit w\"are dann das Endlosschleifen-Problem
gel\"ost.

Der Code dazu befindet sich in:
\begin{mdframed}[style=codebox]
\textbf{Relevanter Ausschnitt unseres Custom Serializers}
\lstinputlisting[language=Java, firstline=30]
{%
  ../code/ms3RESTSpringDemo/src/main/java/%
  ms3restspringdemo/serializers/CustomSetSerializer.java
}
\end{mdframed}

Um den Serializer für die Gerichte einzusetzen, muss eine
entsprechende Annotation mit dem Custom-Serialisierer beim
der Getter-Methode der Gerichte in der Klasse Speise
angebracht werden:

\begin{mdframed}[style=codebox]
\begin{lstlisting}
@JsonProperty
@JsonSerialize(using = CustomSetSerializer.class)
public Set<Gericht> getGerichte() {
	return Collections.unmodifiableSet(gerichte);
}
\end{lstlisting}
\end{mdframed}

% {{{
\begin{mdframed}[style=codebox]
\textbf{Beispielausgabe der Gerichte mit Custom Serializer}
\begin{lstlisting}
[
   {
      "id":2,
      "name":"Schnitzel mit Pommes",
      "details":"Der grandiose Klassiker!",
      "preis":12.5,
      "speisen":[
         {
            "id":2,
            "name":"Schnitzel",
            "gerichte":[
               2
            ]
         },
         {
            "id":1,
            "name":"Pommes",
            "gerichte":[
               3,
               2
            ]
         }
      ]
   },
   {
      "id":3,
      "name":"Grosse Portion Pommes",
      "details":"Lecker fettig!",
      "preis":6,
      "speisen":[
         {
            "id":1,
            "name":"Pommes",
            "gerichte":[
               3,
               2
            ]
         }
      ]
   }
]
\end{lstlisting}
\end{mdframed}
% }}}

Die Ausgabe ist, unserer Meinung nach, viel besser lesbar
und es ist einfacher mit ihr zu arbeiten.

Bei den Speisen sehen wir die Gerichte nun als eine Liste
von IDs.

Nachteil zu der vorigen L\"osung ist, dass Speisen mehrfach
aufgef\"uhrt werden, wie z.B "`Pommes"', weshalb die
Ausgabe mit dem Custom Serializer nicht redundanzfrei ist.

Trozdem ist unserer Meinung nach die Ausgabe des Custom
Serializers besser als die Ausgabe von
"`@JsonIdentityInfo"', da aus unserer Sicht eine wohl-
formatierte Ausgabe wichtiger ist als Redundanzfreiheit.
% }}}

\newpage

% Nachweis der Lauffaehigkeit {{{
\subsubsection{Nachweis der Lauff\"ahigkeit}

\begin{figure}[H]
  \begin{center}
    \includegraphics{ms3/static/postman_collection.PNG}
  \end{center}
  \caption{Postman Collection}
\end{figure}

\begin{figure}[H]
  \begin{center}
    \includegraphics{ms3/static/postman_test.PNG}
  \end{center}
  \caption{Postman Test}
\end{figure}
% }}}

% }}}

\section{Meilenstein 4 $-$ Microservices}

% Teilaufgabe 1 {{{
\subsection{Teilaufgabe 1: Context Map}

\textbf{Anmerkung:} im Folgenden ist der Ausdruck Subsytem
equivalent zum Ausdruck Dom\"ane.

% Map {{{
\subsubsection{Context Map}
\begin{figure}[H]
\begin{center}
\begin{tikzpicture}[
  zub/.style={
    draw,rounded corners,outer color=yellow!25,
    inner color=white
  },
  buc/.style={
    draw, rounded corners, outer color=black!25,
    inner color=white
  },
  rvw/.style={
    draw, rounded corners, outer color=green!25,
    inner color=white
  },
  trv/.style={
    draw, rounded corners, outer color=red!25,
    inner color=white
  },
  lvw/.style={
    draw, rounded corners, outer color=blue!25,
    inner color=white
  }
]
% Zubereitung {{{
\node[zub] at (0,0) (best) {Bestellung};
\node[zub, right=1 of best] (ap) {Arbeitsplatz}
  edge (best);
\node[zub, left=1 of best] (sp) {Sitzplatz}
  edge (best);
\node[zub, below=3 of best] (za) {Zubereitungsanleitung};
\node[zub, left=1 of za] (g) {Gericht}
  edge (za);
\node[zub, left=1 of g] (sk) {Speisekarte}
  edge (g);
\node[zub, right=1 of za] (zp) {Zutatenposition}
  edge (za);
\node[zub, right=1 of zp] (z) {Zutat}
  edge (zp);

\draw (sk) |- (sp);
\draw (g) |- ($(g)!.5!(best)$) -| (best);

\draw[dashed]
  ($(sk.south west) - (1,1)$)
  --
  ($(z.south east) - (-1,1)$)
  |-
  node[below left] {Subsystem Zubereitung}
  ($(best.north) + (0,1)$)
  -|
  cycle;
% }}}

% Buchhaltung {{{
\node[buc, above=3 of best] (buc_best) {Bestellung}
  edge[->,dashed] node[fill=yellow!50] {C/S} (best);

\draw[dashed]
  ($(buc_best.south west)-(1,1)$)
  rectangle
  ($(buc_best.north east)+(2,1)$)
  node[below left] {Subsystem Buchhaltung}
  ;
% }}}

% Tischreservierung {{{
\node[trv,left=4 of buc_best] (trv_sp) {Sitzplatz};

\draw
  (trv_sp)
  |-
  ($(trv_sp)!.5!(sp)$) node[fill=green!75] {SW}
  -|
  (sp);

\draw[dashed]
  ($(trv_sp.north west)-(2,-1)$)
  node[below right] {Subsystem Tischreservierung}
  rectangle
  ($(trv_sp.south east)+(2,-1)$)
  ;
% }}}

% Rezeptverwaltung {{{
\node [rvw, below=3 of sk] (rvw_sk) {Speisekarte}
  edge[<-,dashed] node[fill=yellow!50] {C/S} (sk);
\node [rvw, below=3 of g] (rvw_g) {Gericht}
  edge[<-,dashed] node[fill=yellow!50] {C/S} (g);
\node [rvw, below=3 of za] (rvw_za) {Zubereitungsanleitung}
  edge[<-,dashed] node[fill=yellow!50] {C/S} (za);
\node [rvw, below=3 of zp] (rvw_zp) {Zutatenposition}
  edge[<-,dashed] node[fill=yellow!50] {C/S} (zp);

\draw[dashed]
  ($(rvw_sk.south west) - (1,1)$)
  --
  node[above]{Subsystem Rezeptverwaltung}
  ($(rvw_zp.south east) - (-1,1)$)
  |-
  ($(rvw_za.north) + (0,1)$)
  -|
  cycle;
% }}}

% Lagerverwaltung {{{
\node [lvw, below=6 of z] (lvw_z) {Zutat};

\draw[->,dashed]
  (z) -- node[fill=yellow!50] {C/S} (lvw_z);

\draw[dashed]
  ($(lvw_z.south west) - (4,1)$)
  node[above right]{Subsystem Lagerverwaltung}
  rectangle
  ($(lvw_z.north east) + (1,1)$)
  ;
% }}}

\end{tikzpicture}
\end{center}
\caption{Context Map (logisches Datenmodell)}
\end{figure}

% Legende {{{
\begin{itemize}[label={}]
  \item \tikz[baseline={-3pt}]{
    \node[fill=yellow!50,inner ysep=3pt]
      {C/S};
  }: Customer / Supplier

  \item \tikz[baseline={-3pt}]{
    \node[fill=green!75,inner ysep=3pt]{SW};
  }: Separate Ways

  \item \tikz[baseline={-3pt}]{
    \draw[->,dashed] (0,0) -- (.8,0);
  }: Customer $\rightarrow$ Supplier (Pfeilspitze auf
     Eigent\"umer gerichtet)
\end{itemize}
% }}}

% }}}

% Tabelle {{{
\subsubsection{Tabelle der \"Uberlappungstypen}

\begin{tabu} to \linewidth {p{1.8cm}|p{2.2cm}|X|X}
% Headerzeile {{{
\hline\rowcolor{codebordercolor}
Entity &\"Uberlappung mit anderer Dom\"ane
  &\"Uberlappungstyp &Begr\"undung \\
% }}}
% Bestellung {{{
\hline
Bestellung &Buchhaltung &Conformist (unser Subsystem als
  Eigent\"umer) &Die Buchhaltung ruft die Bestellungsdaten
  bei uns ab. Hier wurde Conformist anstelle von Customer /
  Suplier gew\"ahlt, da es sich bei dem Buchhaltungssystem
  wahrscheinlich nicht um eine Hausentwicklung handelt,
  sondern um ein propriet\"ares System mit unbekannten
  Schnittstellen, weshalb eine Zusammenarbeit auf
  Augenh\"ohe nicht unbedingt m\"oglich ist. \\
% }}}
% Gericht {{{
\hline
Gericht &Rezeptver\-waltung &Customer / Suplier
  (Rezeptverwaltung als Eigent\"umer) &Wir rufen
  die Gerichte beim Subsystem Rezeptverwaltung ab. Das
  Subsystem Rezeptverwaltung ist der Eigent\"umer und wir
  haben (brauchen) nur lesenden Zugriff auf das Entity
  Gericht. Customer / Suplier, da wir auf Augenh\"ohe mit
  der Rezeptverwaltung sind uns eine enge Zusammenarbeit
  m\"oglich ist. \\
% }}}
% Speisekarte {{{
\hline
Speisekar\-te &Rezeptver\-waltung &Customer / Suplier
  (Rezeptverwaltung als Eigent\"umer) &vgl. Entity Gericht.
  \\
% }}}
% Zubereitungsanleitung {{{
\hline
Zuberei\-tungsanlei\-tung &Rezeptver\-waltung &
  Customer / Suplier (Rezeptverwaltung als Eigent\"umer)
  &vgl. Entity Gericht. \\
% }}}
% Zutat {{{
\hline
Zutat &Lagerver\-waltung &Customer / Suplier
  (Lagerverwaltung als Eigent\"umer) &Zutat ist in unserem
  Fall einfach die Menge der Zutat, welche im Lager
  zur Verf\"ugung steht und wird mit der Lagerverwaltung
  abgeglichen. Zutat verh\"alt sich analog zu Gericht. \\
% }}}
% Zutatenposition {{{
\hline
Zutaten\-position &Rezeptver\-waltung &Customer / Suplier
  (Rezeptverwaltung als Eigent\"umer) &vgl. Entity Gericht.
  \\
% }}}
\hline
\end{tabu}
% }}}
% }}}

% Teilaufgabe 2 {{{
\subsection{Teilaufgabe 2: Aggregates}

Im Folgenden beziehen wir uns auf unsere Aggregates aus
Kapitel \ref{ms3_aggregates}. Die einzige Unterscheidung zu
diesem Aggregate ist, dass wir die redundante Entity Speise
(in Kapitel \ref{ldm} dem Klassendiagramm des logischen
Datenmodells hinzugef\"ugt, um die Aufgabenstellung zu
erf\"ullen) entfernen und das Attribut $Speise.name$ in
$Zubereitungsanleitung.name$ \"uberf\"uhren.

\begin{center}
% Tabelle {{{
\begin{tabu} to \linewidth {p{3cm}|X|X|X}
% Headerzeile {{{
\hline\rowcolor{codebordercolor}
Aggregate Root &Weitere beteiligte Entities &Invarianten
  &Begr\"undung, dass das ein Aggregate ist \\
% }}}
% Gericht {{{
\hline
Gericht &Zubereitungsanleitung, Zutatenposition, Zutat
  &\textit{Gericht.name} wird aus
  \textit{Zubereitungsanleitung.name}
  zusammengesetzt (Schnitzel, Pommes, Salat $\Rightarrow$
  \textit{Gericht.name} = Schnitzel mit Pommes und Salat)
  &vgl. \ref{ms3_aggregates} \\
% }}}
\hline
\end{tabu}
% }}}
\end{center}
% }}}

\subsection{Teilaufgabe 3: Microservice-Architektur}

\subsubsection{Servicetabelle}

% Tabelle {{{
\begin{tabu} to \linewidth {X|X|X}
% Headerzeile {{{
\hline\rowcolor{codebordercolor}
Service &Bildet ab &Kommentar \\
% }}}
% Bestellung {{{
\hline
Bestellungsdaten &Bestellung &Dient als API Gateway, da
  unser Subsystem ein Supplier f\"ur die Bestelldaten ist.
  \\
% }}}
% Gericht {{{
\hline
Gerichtsdaten &Gericht (vgl. \ref{ms3_aggregates})
  &Adapterservice, da unser Subsystem ein Customer f\"ur
  die Gerichtsdaten ist.\\
% }}}
% Speisekarte {{{
\hline
Speisekartendaten &Speisekarte &Adapterservice, da unser
  Subsystem ein Customer f\"ur die Speisekartendaten ist.\\
% }}}
% Restaurant {{{
\hline
Restaurantdaten &Sitzplatz, Arbeitsplatz &Service der die
  Sitz- und Arbeitspl\"atze der Standorte verwaltet. \\
% }}}
% Gast-UI {{{
\hline
Gast-UI & &UI-Service f\"ur den Gast. Hier wird unserer
  Meinung nach kein API Gateway ben\"otigt, da das UI nur
  Daten aus dem Speisekartendaten-Service ben\"otigt und
  Bestellungen an den Bestellungsdaten-Service schickt,
  was weder eine besondere Darstellung der Daten ist, noch
  als viele verschiedene Aufrufe zu charakterisieren
  w\"are. \\
% }}}
% Koch-UI {{{
\hline
Koch-UI & &UI-Service f\"ur den Koch. Hier wird unserer
  Meinung nach kein API Gateway ben\"otigt, da nur Daten
  aus dem Gerichtsdaten-Service abgerufen werden m\"ussen.
  \\
% }}}
\hline
\end{tabu}
% }}}

\subsubsection{Komponentendiagramm}
\begin{figure}[H]
\begin{center}
\begin{tikzpicture} [n_api/.style={
  inner ysep=0cm, inner xsep=.2cm
}]
  % Compos {{{
  \UMLSComponent
    {0}
    {0}
    {Gast-UI}
    {fill=green!25,text width=3cm}

    \UMLSComponentRelativeTo
      {below=2 of Gast-UI}
      {Speisekartendaten}
      {fill=black!25,text width=3cm}

    % DUMMY NODE
    \node[
      right=2 of Gast-UI,
      text width=3cm,
      minimum height=1.4cm
    ] (Bestellabwicklung) {};

      \UMLSComponentRelativeTo
        {above=2 of Bestellabwicklung}
        {Bestellungsdaten}
        {fill=purple,text width=3cm}

        \UMLSComponentRelativeToAlterName
          {above=2 of Bestellungsdaten}
          {$<<$Buchhal\-tung$>>$}
          {bn}
          {fill=black!50,text width=3cm}

      \UMLSComponentRelativeTo
        {right=2 of Bestellabwicklung}
        {Koch-UI}
        {fill=green!25,text width=3cm}

        \UMLSComponentRelativeTo
          {below=2 of Koch-UI}
          {Gerichtsdaten}
          {fill=black!25,text width=3cm}

          \UMLSComponentRelativeToAlterName
            {below=2 of Gerichtsdaten}
            {$<<$Lagerver\-waltung$>>$}
            {lvw}
            {fill=black!50,text width=3cm}

      \UMLSComponentRelativeTo
        {below=2 of Bestellabwicklung}
        {Restaurantdaten}
        {fill=white,text width=3cm}

        \UMLSComponentRelativeToAlterName
          {below=2 of Restaurantdaten}
          {$<<$Rezeptver\-waltung$>>$}
          {rvw}
          {fill=black!50,text width=3cm}
  % }}}

  % Conns {{{

    % Providers {{{

      % Bestellungsdaten {{{
      \draw [-umlportproviderREST]
        (Bestellungsdaten)
        --
        ($(Bestellungsdaten)!.5!(bn)$)
          node[n_api] (bdr) {};

      \draw [-umlportproviderEvent]
        (Bestellungsdaten) -- (Bestellabwicklung.center);
      % }}}

      % Speisekartendaten {{{
      \draw [-umlportproviderREST]
        (Speisekartendaten)
        --
        ($(Speisekartendaten)!.5!(Gast-UI)$)
          node[n_api] (skr) {};
      % }}}

      % Restaurantdaten {{{
      \draw [-umlportproviderREST]
        (Restaurantdaten)
        --
        ($(Restaurantdaten)!.5!(Bestellabwicklung)$)
          node[n_api] (rdr) {};
      % }}}

      % Gerichtsdaten {{{
      \draw [-umlportproviderREST]
        (Gerichtsdaten)
        --
        ($(Gerichtsdaten)!.5!(Koch-UI)$)
          node[n_api] (gdr) {};
      % }}}

      % Rezeptverwaltung {{{
      \draw [-umlportproviderREST]
        (rvw)
        --
        ($(rvw)!.5!(Restaurantdaten)$)
          node[n_api] (rvr) {};
      % }}}

      % Lagerverwaltung {{{
      \draw [-umlportproviderREST]
        (lvw)
        --
        ($(lvw)!.5!(Gerichtsdaten)$)
          node[n_api] (lvr) {};
      % }}}

    % }}}

    % Customers {{{

      % Gast-UI {{{
      \draw [-umlportcaller] (Gast-UI) -- (skr);

      \draw [-umlportcaller]
        (Gast-UI.4)
        --
        ($(Bestellabwicklung.center)+(-.2,.15)$);

      \draw [-umlportcaller]
        (Gast-UI.350)
        -|
        ($(Gast-UI)!.5!(Restaurantdaten)$)
        |-
        ($(rdr.west)-(0,.15)$);

      \draw [-umlportcaller]
        (Gast-UI)
        |-
        ($(bdr.west)-(0,.15)$);
      % }}}

      % Koch-UI {{{
      \draw [-umlportcaller] (Koch-UI) -- (gdr);

      \draw [-umlportcaller]
        (Koch-UI.176)
        --
        ($(Bestellabwicklung.center)+(.2,.15)$);

      \draw [-umlportcaller]
        (Koch-UI.190)
        -|
        ($(Koch-UI)!.5!(Restaurantdaten)$)
        |-
        ($(rdr.east)-(0,.15)$);

      \draw [-umlportcaller]
        (Koch-UI)
        |-
        ($(bdr.east)-(0,.15)$);
      % }}}

      % Buchhaltung {{{
      \draw [-umlportcaller] (bn) -- (bdr);
      % }}}

      % Gerichtsdaten {{{
      \draw [-umlportcaller] (Gerichtsdaten) -- (lvr);

      \draw [-umlportcaller]
        (Gerichtsdaten.220)
        |-
        ($(rvr.east)-(0,.15)$);
      % }}}

      % Speisekartendaten {{{
      \draw [-umlportcaller]
        (Speisekartendaten)
        |-
        ($(rvr.west)-(0,.15)$);
      % }}}

    % }}}

  % }}}
\end{tikzpicture}
\caption{Komponentendiagramm}
\end{center}
\end{figure}

% Legende {{{
\begin{itemize}[label={}]
  \item \tikz[baseline={-3pt}]{
    \node[fill=white,draw,inner ysep=3pt]
      {\textcolor{white}{X}};
  }: Service

  \item \tikz[baseline={-3pt}]{
    \node[fill=green!25,green!25,inner ysep=3pt]{X};
  }: UI-Service

  \item \tikz[baseline={-3pt}]{
    \node[fill=purple,purple,inner ysep=3pt]{X};
  }: API Gateway

  \item \tikz[baseline={-3pt}]{
    \node[fill=black!25,black!25,inner ysep=3pt]{X};
  }: Adapterservice

  \item \tikz[baseline={-3pt}]{
    \node[fill=black!50,black!50,inner ysep=3pt]{X};
  }: Umsystem

\item \tikz[baseline={-3pt}]{
    \node[fill=blue,blue,shape=circle,inner sep=0pt]{X};
  }: REST-Api

\item \tikz[baseline={-3pt}]{
    \node[fill=red,red,shape=circle,inner sep=0pt]{X};
  }: Event-Api
\end{itemize}
% }}}


\begin{appendices}

% Postman Logs {{{
\section{Postman Logs}

% Postman-Collection {{{
\subsection{Postman-Collection}
\lstinputlisting[
  language=Java,
]{%
  ms3/static/st-ms3.%
  postman_collection.json
}
% }}}

% Postman-Test-Run {{{
\subsection{Postman-Test-Run}
\lstinputlisting[
  language=Java,
]{%
  ms3/static/st-ms3.%
  postman_test_run.json
}
% }}}

% }}}

\end{appendices}


\end{document}
