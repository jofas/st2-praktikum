\documentclass[12pt, twoside]{article}
\usepackage{amsmath}
\usepackage{amssymb}
\usepackage[colorlinks=true, linkcolor=black]{hyperref} % Links
\usepackage{makeidx} % Indexierung
\usepackage{siunitx}
\usepackage[ngerman]{babel} % deutsche Sonderzeichen
\usepackage[utf8]{inputenc}
\usepackage{geometry} % Dokumentendesign wie Seiten- oder Zeilenabstand bestimmen

% Graphiken
\usepackage{tikz}
\usepackage{pgfplots}
\usepackage{pgfcore}
\usepackage{pgfopts}
\usepackage{pgf}
\usepackage{ifthen}
\usepackage{booktabs}

% Tabellen
\usepackage{tabu}
\usepackage{longtable}
\usepackage{colortbl} % Tabellen faerben
\usepackage{multirow}
\usepackage{diagbox} % Tabellenzelle diagonal splitten

\usepackage{xcolor} % Farben
\usepackage[framemethod=tikz]{mdframed} % Hintergrunderstellung
\usepackage{enumitem} % Enumerate mit Buchstaben nummerierbar machen
\usepackage{pdfpages}
\usepackage{listings} % Source-Code darstellen
\usepackage{eurosym} % Eurosymbol
\usepackage[square,numbers]{natbib}
\usepackage{here} % figure an richtiger Stelle positionieren
\usepackage{verbatim} % Blockkommentare mit \begin{comment}...\end{comment}
\usepackage{ulem} % \sout{} (durchgestrichener Text)

% BibLaTex
\bibliographystyle{acm}

% mdframed Style
\mdfdefinestyle{codebox}{
	linewidth=2.5pt,
	linecolor=codebordercolor,
	backgroundcolor=codecolor,
	shadow=true,
	shadowcolor=black!40!white,
	fontcolor=black,
	everyline=true,
}

% Seitenabstaende
\geometry{left=15mm,right=15mm,top=15mm,bottom=20mm}

% TikZ Bibliotheken
\usetikzlibrary{
    arrows,
    arrows.meta,
    decorations,
    backgrounds,
    positioning,
    fit,
    petri,
    shadows,
    datavisualization.formats.functions,
    calc,
    shapes.multipart
}

\pgfplotsset{width=7cm,compat=1.15}

\definecolor{codecolor}{HTML}{EEEEEE}
\definecolor{codebordercolor}{HTML}{CCCCCC}

% Standardeinstellungen fuer Source-Code
\lstset{
    language=C,
    breaklines=true,
    keepspaces=true,
    keywordstyle=\bfseries\color{green!70!black},
    basicstyle=\ttfamily\color{black},
    commentstyle=\itshape\color{purple},
    identifierstyle=\color{blue},
    stringstyle=\color{orange},
    showstringspaces=false,
    rulecolor=\color{black},
    tabsize=2,
    escapeinside={\%*}{*\%},
}

%%%%%%%%%%%%%%%%%%%%%%%%%%%%%%%%%%%%%%%%%%%%%%%%%%%%%%%%%%%%%%%%%%%%%%%%%%%%%%%%%%%%%%%%%%%%%%%%%%%%%%%%%%%%%%%%%%%%%%%%%%%%%%%%%%%%%%%%%%%%%%%%%%
%
%UML Module
%
%%%%%%%%%%%%%%%%%%%%%%%%%%%%%%%%%%%%%%%%%%%%%%%%%%%%%%%%%%%%%%%%%%%%%%%%%%%%%%%%%%%%%%%%%%%%%%%%%%%%%%%%%%%%%%%%%%%%%%%%%%%%%%%%%%%%%%%%%%%%%%%%%%

% arrow tips {{{
\tikzset{
  umlaggreg /.tip = {Diamond[fill=white,scale=1.75]}
}
\tikzset{
  umlcompo /.tip = {Diamond[fill=black,scale=1.75]}
}
% }}}

%%%%%%%%%%%%%%%%%%%%%%%%%%%%%%%%%%%%%%%%%%%%%%%%%%%%%%%%%%%%%%%%%%%%%%%%%%%%%%%%%%%%%%%%%%%%%%%%%%%%%%%%%%%%%%%%%%%%%%%%%%%%%%%%%%%%%%%%%%%%%%%%%%
%UML Component Token
%%%%%%%%%%%%%%%%%%%%%%%%%%%%%%%%%%%%%%%%%%%%%%%%%%%%%%%%%%%%%%%%%%%%%%%%%%%%%%%%%%%%%%%%%%%%%%%%%%%%%%%%%%%%%%%%%%%%%%%%%%%%%%%%%%%%%%%%%%%%%%%%%%
\def\UMLComponentToken#1#2#3{
	\draw [#3]($(#1.north east)-(0.1,0.1)$) rectangle ($(#1.north east)-(0.3,0.45)$);
	\draw [#3,fill=#2]($(#1.north east)-(0.35,0.2)$) rectangle ($(#1.north east)-(0.25,0.25)$);
	\draw [#3,fill=#2]($(#1.north east)-(0.35,0.3)$) rectangle ($(#1.north east)-(0.25,0.35)$);
}

%%%%%%%%%%%%%%%%%%%%%%%%%%%%%%%%%%%%%%%%%%%%%%%%%%%%%%%%%%%%%%%%%%%%%%%%%%%%%%%%%%%%%%%%%%%%%%%%%%%%%%%%%%%%%%%%%%%%%%%%%%%%%%%%%%%%%%%%%%%%%%%%%%
%UML Component
%%%%%%%%%%%%%%%%%%%%%%%%%%%%%%%%%%%%%%%%%%%%%%%%%%%%%%%%%%%%%%%%%%%%%%%%%%%%%%%%%%%%%%%%%%%%%%%%%%%%%%%%%%%%%%%%%%%%%%%%%%%%%%%%%%%%%%%%%%%%%%%%%%
%USAGE:
% 	\UMLComponent{name}{width}{height}{Position eines Ports und Beschreibender UMLNote in der Form:
%																				positionPort/noteX/noteY/noteText/noteColor
%
%BEISPIEL:
%		\UMLComponent{Bestellvorgang}{\linewidth-0.2cm}{15cm}{
%			Bestellvorgang.east/7/1.5/Bestellvorgang ausf\"uhren/yellow!25,
%			Bestellvorgang.south/-2/-6/Blah/yellow!25,
%			Bestellvorgang.310/7/-6/Blahhh/yellow!25
%		}
\def\UMLComponent#1#2#3#4{
	\node [rectangle split, rectangle split parts = 2,rectangle split empty part height=#3,draw,fill=yellow!25,inner ysep=7pt,inner xsep=15pt,minimum size=#2] at(0,0) (#1) {#1 \nodepart{second}};
	\UMLComponentToken{#1}{yellow!25}{}
	\foreach \port/\notex/\notey/\notetext/\notecolor in {#4}{
		\UMLNote{\notex}{\notey}{\notetext}{\port}{\notecolor};
		\draw[fill=yellow!25] ($(\port) - (.2,.2)$) rectangle ($(\port)+(.2,.2)$);
	};
}

%%%%%%%%%%%%%%%%%%%%%%%%%%%%%%%%%%%%%%%%%%%%%%%%%%%%%%%%%%%%%%%%%%%%%%%%%%%%%%%%%%%%%%%%%%%%%%%%%%%%%%%%%%%%%%%%%%%%%%%%%%%%%%%%%%%%%%%%%%%%%%%%%%
%UML Simple Component
%%%%%%%%%%%%%%%%%%%%%%%%%%%%%%%%%%%%%%%%%%%%%%%%%%%%%%%%%%%%%%%%%%%%%%%%%%%%%%%%%%%%%%%%%%%%%%%%%%%%%%%%%%%%%%%%%%%%%%%%%%%%%%%%%%%%%%%%%%%%%%%%%%
\def\UMLSComponent#1#2#3{
	\node [rectangle split, rectangle split parts = 2,rectangle split empty part height=1cm,draw, fill=yellow!25,inner ysep=7pt,inner xsep=15pt] at(#1,#2) (#3) {#3 \nodepart{second}};
	\UMLComponentToken{#3}{yellow!25}{}
}

% 1. Relative to
% 2. Text/Name
% 3. Options
\def\UMLSComponentRelativeTo#1#2#3{
	\node [rectangle split, rectangle split parts = 2,rectangle split empty part height=1cm,draw, fill=yellow!25,inner ysep=7pt,inner xsep=15pt,#1,#3] (#2) {#2 \nodepart{second}};
	\UMLComponentToken{#2}{yellow!25}{}
}

% 1. Relative to
% 2. Text
% 3. Name
% 4. Options
\def\UMLSComponentRelativeToAlterName#1#2#3#4{
	\node [rectangle split, rectangle split parts = 2,rectangle split empty part height=1cm,draw, fill=yellow!25,inner ysep=7pt,inner xsep=15pt,#1,#4](#3) {#2 \nodepart{second}};
	\UMLComponentToken{#3}{yellow!25}{}
}
%%%%%%%%%%%%%%%%%%%%%%%%%%%%%%%%%%%%%%%%%%%%%%%%%%%%%%%%%%%%%%%%%%%%%%%%%%%%%%%%%%%%%%%%%%%%%%%%%%%%%%%%%%%%%%%%%%%%%%%%%%%%%%%%%%%%%%%%%%%%%%%%%%
%UML Extern Component
%%%%%%%%%%%%%%%%%%%%%%%%%%%%%%%%%%%%%%%%%%%%%%%%%%%%%%%%%%%%%%%%%%%%%%%%%%%%%%%%%%%%%%%%%%%%%%%%%%%%%%%%%%%%%%%%%%%%%%%%%%%%%%%%%%%%%%%%%%%%%%%%%%
\def\UMLExternComponent#1#2#3{
	\node [rectangle split, rectangle split parts = 2,rectangle split empty part height=1cm,draw, fill=black!10,inner ysep=7pt,inner xsep=15pt] at(#1,#2) (#3) {$<<$#3$>>$ \nodepart{second}};
	\UMLComponentToken{#3}{black!10}{}
}

%%%%%%%%%%%%%%%%%%%%%%%%%%%%%%%%%%%%%%%%%%%%%%%%%%%%%%%%%%%%%%%%%%%%%%%%%%%%%%%%%%%%%%%%%%%%%%%%%%%%%%%%%%%%%%%%%%%%%%%%%%%%%%%%%%%%%%%%%%%%%%%%%%
%UML Note
%%%%%%%%%%%%%%%%%%%%%%%%%%%%%%%%%%%%%%%%%%%%%%%%%%%%%%%%%%%%%%%%%%%%%%%%%%%%%%%%%%%%%%%%%%%%%%%%%%%%%%%%%%%%%%%%%%%%%%%%%%%%%%%%%%%%%%%%%%%%%%%%%%
%USAGE:
%	\UMLNote{x}{y}{text}{to (--)}{color (rectangle north east)}
\def\UMLNote#1#2#3#4#5{
	\node [rectangle,fill=green!25,draw,inner ysep=15pt,inner xsep=6pt,text width=3cm] at (#1,#2) (b) {\small{#3}};
	\draw [fill=#5,#5]($(b.north east) + (.1,.1)$) -- ($(b.north east) - (0,.5)$) -- ($(b.north east) - (.5,0)$) -- cycle;
	\draw ($(b.north east) - (.01,.5)$) -- ($(b.north east) - (.5,.01)$) -- ($(b.north east)-(.5,.5)$) -- cycle;
	\draw [dashed](b) -- (#4);
}

%%%%%%%%%%%%%%%%%%%%%%%%%%%%%%%%%%%%%%%%%%%%%%%%%%%%%%%%%%%%%%%%%%%%%%%%%%%%%%%%%%%%%%%%%%%%%%%%%%%%%%%%%%%%%%%%%%%%%%%%%%%%%%%%%%%%%%%%%%%%%%%%%%
% UML Component Port
%%%%%%%%%%%%%%%%%%%%%%%%%%%%%%%%%%%%%%%%%%%%%%%%%%%%%%%%%%%%%%%%%%%%%%%%%%%%%%%%%%%%%%%%%%%%%%%%%%%%%%%%%%%%%%%%%%%%%%%%%%%%%%%%%%%%%%%%%%%%%%%%%%
\def\UMLComponentPort#1{
	\draw[fill=yellow!25] ($(#1)-(.2,.2)$) rectangle ($(#1)+(.2,.2)$);
}

%%%%%%%%%%%%%%%%%%%%%%%%%%%%%%%%%%%%%%%%%%%%%%%%%%%%%%%%%%%%%%%%%%%%%%%%%%%%%%%%%%%%%%%%%%%%%%%%%%%%%%%%%%%%%%%%%%%%%%%%%%%%%%%%%%%%%%%%%%%%%%%%%%
% UML Component Connector with Ports on both ends
%%%%%%%%%%%%%%%%%%%%%%%%%%%%%%%%%%%%%%%%%%%%%%%%%%%%%%%%%%%%%%%%%%%%%%%%%%%%%%%%%%%%%%%%%%%%%%%%%%%%%%%%%%%%%%%%%%%%%%%%%%%%%%%%%%%%%%%%%%%%%%%%%%
\def\UMLComponentPortConnector#1#2#3#4#5#6#7#8{
	\draw[#3,-{>[sep=3.5]>}] (#1) #4 (#2);
	\UMLNote{#5}{#6}{#7}{#2}{#8}
	\draw[fill=yellow!25] ($(#1)-(.2,.2)$) rectangle ($(#1)+(.2,.2)$);
	\draw[fill=yellow!25] ($(#2)-(.2,.2)$) rectangle ($(#2)+(.2,.2)$);
}

%%%%%%%%%%%%%%%%%%%%%%%%%%%%%%%%%%%%%%%%%%%%%%%%%%%%%%%%%%%%%%%%%%%%%%%%%%%%%%%%%%%%%%%%%%%%%%%%%%%%%%%%%%%%%%%%%%%%%%%%%%%%%%%%%%%%%%%%%%%%%%%%%%
% UML Component Realizor with Ports on both ends
%%%%%%%%%%%%%%%%%%%%%%%%%%%%%%%%%%%%%%%%%%%%%%%%%%%%%%%%%%%%%%%%%%%%%%%%%%%%%%%%%%%%%%%%%%%%%%%%%%%%%%%%%%%%%%%%%%%%%%%%%%%%%%%%%%%%%%%%%%%%%%%%%%
\def\UMLComponentPortRealizor#1#2#3#4#5#6#7#8{
	\draw[#3,-{Stealth[sep=3.5,inset=0pt,length=5pt,width=5pt]>},blue] (#1) #4 (#2);
	\UMLNote{#5}{#6}{#7}{#2}{#8}
	\draw[fill=yellow!25] ($(#1)-(.2,.2)$) rectangle ($(#1)+(.2,.2)$);
	\draw[fill=yellow!25] ($(#2)-(.2,.2)$) rectangle ($(#2)+(.2,.2)$);
}

% UML Class (various) {{{
%
% 1. Normal                         -> \UMLClass { X } { Y } { Text/Name } { Options (node) }
% 2. Normal mit alternativem Namen  -> \UMLClassAlterName { X } { Y } { Text } { Name } { Options (node) }
% 3. Relativ                        -> \UMLClassRelativeTo { Relative to } { Text/Name } { Options (node) }
% 4. Relativ mit alternativem Namen -> \UMLClassRelativeToAlterName { Relative to } { Text } { Name } { Options (node) }

  % 1. Normal {{{
  \def\UMLClass#1#2#3#4{
    \node [
      fill = yellow!25,
      rectangle split,
      rectangle split parts = 3,
      rectangle split ignore empty parts = true,
      rectangle split part align = {center, left, left},
      inner xsep = 15pt,
      inner ysep = 7pt,
      draw,
      #4
    ] at(#1,#2) (#3) {#3};
  }
  % }}}

  % 2. Normal mit alternativem Namen {{{
    \def\UMLClassAlterName#1#2#3#4#5{
      \node [
        fill = yellow!25,
        rectangle split,
        rectangle split parts = 3,
        rectangle split ignore empty parts = true,
        rectangle split part align = {center, left, left},
        inner xsep = 15pt,
        inner ysep = 7pt,
        draw,
        #5
      ] at (#1, #2) (#4) {#3};
    }
  % }}}

  % 3. Relativ {{{
    \def\UMLClassRelativeTo#1#2#3{
      \node [
        fill = yellow!25,
        rectangle split,
        rectangle split parts = 3,
        rectangle split ignore empty parts = true,
        rectangle split part align = {center, left, left},
        inner xsep = 15pt,
        inner ysep = 7pt,
        draw,
        #1,
        #3
      ] (#2) {#2};
    }
  % }}}

  % 4. Relativ {{{
    \def\UMLClassRelativeToAlterName#1#2#3#4{
      \node [
        fill = yellow!25,
        rectangle split,
        rectangle split parts = 3,
        rectangle split ignore empty parts = true,
        rectangle split part align = {center, left, left},
        inner xsep = 15pt,
        inner ysep = 7pt,
        draw,
        #1,
        #4
      ] (#3) {#2};
    }
  % }}}

% }}}

%
% UML Activity State (various)
%
% 1. Normal                         -> \UMLActivityState{ X }{ Y }{ Text/Name }{ Options (node) }
% 2. Normal mit alternativem Namen  -> \UMLActivityStateAlterName{ X }{ Y }{ Text }{ Name }{ Options (node) }
% 3. Relativ                        -> \UMLActivityStateRelativeTo{ Relative to }{ Text/Name }{ Options (node) }
% 4. Relativ mit alternativem Namen -> \UMLActivityStateRelativeToAlterName{ Relative to }{ Text }{ Name }{ Options (node) }
%
%

% 1. Normal
%
% #1: X
% #2: Y
% #3: Text/Name
% #4: Options (node)

\def\UMLActivityState#1#2#3#4{
	\node [rectangle,rounded corners,draw,fill=black!10,inner ysep=9pt,inner xsep=22pt,#4] at(#1,#2) (#3) {#3};
	\UMLActivityStateToken{#3}
}

% 2. Alternativer Name
%
% #1: X
% #2: Y
% #3: Text
% #4: Name
% #5: Options (node)

\def\UMLActivityStateAlterName#1#2#3#4#5{
	\node [rectangle,rounded corners,draw,fill=black!10,inner ysep=9pt,inner xsep=22pt,#5] at(#1,#2) (#4) {#3};
	\UMLActivityStateToken{#4}
}


% 3. Relativ
%
% #1: Relative to (z.B. below=1 of ...)
% #2: Text/Name
% #3: Options (node)

\def\UMLActivityStateRelativeTo#1#2#3{
	\node [rectangle,rounded corners,draw,fill=black!10,inner ysep=9pt,inner xsep=22pt,#1,#3] (#2) {#2};
	\UMLActivityStateToken{#2}
}


% 4. Relativ mit alternativem Namen
%
% #1: Relative to (z.B. below=1 of ...)
% #2: Text
% #3: Name
% #4: Options

\def\UMLActivityStateRelativeToAlterName#1#2#3#4{
	\node [rectangle,rounded corners,draw,fill=black!10,inner ysep=9pt,inner xsep=22pt,#1,#4] (#3) {#2};
	\UMLActivityStateToken{#3}
}


% Token

\def\UMLActivityStateToken#1{
	\draw[fill=yellow!25,rounded corners] ($(#1.west)+(.3,.25)$) rectangle ($(#1.west)+(.7,-.15)$);
}


%
% UML Activity Object (various)
%
% 1. Normal                         -> \UMLActivityObject{ X }{ Y }{ Text/Name }{ Options (node) }
% 2. Normal mit alternativem Namen  -> \UMLActivityObjectAlterName{ X }{ Y }{ Text }{ Name }{ Options (node) }
% 3. Relativ                        -> \UMLActivityObjectRelativeTo{ Relative to }{ Text/Name }{ Options (node) }
% 4. Relativ mit alternativem Namen -> \UMLActivityObjectRelativeToAlterName{ Relative to }{ Text }{ Name }{ Options (node) }
%
%

% 1. Normal
%
% #1: X
% #2: Y
% #3: Text/Name
% #4: Options (node)

\def\UMLActivityObject#1#2#3#4{
	\node [rectangle,draw,fill=green!15,inner ysep=9pt,inner xsep=25pt,#4] at(#1,#2) (#3) {:#3};
	\UMLActivityObjectToken{#3}
}


% 2. Alternativer Name
%
% #1: X
% #2: Y
% #3: Text
% #4: Name
% #5: Options (node)

\def\UMLActivityObjectAlterName#1#2#3#4#5{
	\node [rectangle,draw,fill=green!15,inner ysep=9pt,inner xsep=25pt,#5] at(#1,#2) (#4) {:#3};
	\UMLActivityObjectToken{#4}
}


% 3. Relativ
%
% #1: Relative to (z.B. below=1 of ...)
% #2: Text/Name
% #3: Options (node)

\def\UMLActivityObjectRelativeTo#1#2#3{
	\node [rectangle,draw,fill=green!15,inner ysep=9pt,inner xsep=25pt,#3,#1] (#2) {:#2};
	\UMLActivityObjectToken{#2}
}


% 4. Relativ mit alternativem Namen
%
% #1: Relative to (z.B. below=1 of ...)
% #2: Text
% #3: Name
% #4: Options

\def\UMLActivityObjectRelativeToAlterName#1#2#3#4{
	\node [rectangle,draw,fill=green!15,inner ysep=9pt,inner xsep=25pt,#4,#1] (#3) {:#2};
	\UMLActivityObjectToken{#3}
}


% Token

\def\UMLActivityObjectToken#1{
	\draw[fill=yellow!25] ($(#1.west)+(.3,.25)$) rectangle ($(#1.west)+(.7,.11)$);
	\draw ($(#1.west)+(.35,.18)$) -- ($(#1.west)+(.65,.18)$);
	\draw[fill=yellow!25] ($(#1.west)+(.3,.11)$) rectangle ($(#1.west)+(.7,-.02)$);
	\draw[fill=yellow!25] ($(#1.west)+(.3,-.02)$) rectangle ($(#1.west)+(.7,-.15)$);
}


%
% UML Activity Data Storage (various)
%
% 1. Normal                         -> \UMLActivityDataStorage{ X }{ Y }{ Text/Name }{ Options (node) }
% 2. Normal mit alternativem Namen  -> \UMLActivityDataStorageAlterName{ X }{ Y }{ Text }{ Name }{ Options (node) }
% 3. Relativ                        -> \UMLActivityDataStorageRelativeTo{ Relative to }{ Text/Name }{ Options (node) }
% 4. Relativ mit alternativem Namen -> \UMLActivityDataStorageRelativeToAlterName{ Relative to }{ Text }{ Name }{ Options (node) }
%
%

% 1. Normal
%
% #1: X
% #2: Y
% #3: Text/Name
% #4: Options (node)

\def\UMLActivityDataStorage#1#2#3#4{
	\node [rectangle,draw,fill=green!15,inner ysep=9pt,inner xsep=25pt,#4] at(#1,#2) (#3) {#3};
	\UMLActivityDataStorageToken{#3}
}


% 2. Alternativer Name
%
% #1: X
% #2: Y
% #3: Text
% #4: Name
% #5: Options (node)

\def\UMLActivityDataStorageAlterName#1#2#3#4#5{
	\node [rectangle,draw,fill=green!15,inner ysep=9pt,inner xsep=25pt,#5] at(#1,#2) (#4) {#3};
	\UMLActivityDataStorageToken{#4}
}


% 3. Relativ
%
% #1: Relative to (z.B. below=1 of ...)
% #2: Text/Name
% #3: Options (node)

\def\UMLActivityDataStorageRelativeTo#1#2#3{
	\node [rectangle,draw,fill=green!15,inner ysep=9pt,inner xsep=25pt,#3,#1] (#2) {#2};
	\UMLActivityDataStorageToken{#2}
}


% 4. Relativ mit alternativem Namen
%
% #1: Relative to (z.B. below=1 of ...)
% #2: Text
% #3: Name
% #4: Options

\def\UMLActivityDataStorageRelativeToAlterName#1#2#3#4{
	\node [rectangle,draw,fill=green!15,inner ysep=9pt,inner xsep=25pt,#4,#1] (#3) {#2};
	\UMLActivityDataStorageToken{#3}
}


% Token

\def\UMLActivityDataStorageToken#1{
	\node[rectangle,draw,scale=.5,fill=yellow!25] at($(#1.west)+(.5,-.05)$) (ds) {DS};
	\draw[-{>[scale width=.5]}] ($(ds.north west)+(0,.15)$) -| (ds.north);
	\draw[-{>[scale width=.5]}] (ds.east) -- ($(ds.east)+(.15,0)$);
}


%
% UML Activity Central Buffer (various)
%
% 1. Normal                         -> \UMLActivityCentralBuffer{ X }{ Y }{ Text/Name }{ Options (node) }
% 2. Normal mit alternativem Namen  -> \UMLActivityCentralBufferAlterName{ X }{ Y }{ Text }{ Name }{ Options (node) }
% 3. Relativ                        -> \UMLActivityCentralBufferRelativeTo{ Relative to }{ Text/Name }{ Options (node) }
% 4. Relativ mit alternativem Namen -> \UMLActivityCentralBufferRelativeToAlterName{ Relative to }{ Text }{ Name }{ Options (node) }
%
%

% 1. Normal
%
% #1: X
% #2: Y
% #3: Text/Name
% #4: Options (node)

\def\UMLActivityCentralBuffer#1#2#3#4{
	\node [rectangle,draw,fill=green!15,inner ysep=9pt,inner xsep=25pt,#4] at(#1,#2) (#3) {#3};
	\UMLActivityCentralBufferToken{#3}
}


% 2. Alternativer Name
%
% #1: X
% #2: Y
% #3: Text
% #4: Name
% #5: Options (node)

\def\UMLActivityCentralBufferAlterName#1#2#3#4#5{
	\node [rectangle,draw,fill=green!15,inner ysep=9pt,inner xsep=25pt,#5] at(#1,#2) (#4) {#3};
	\UMLActivityCentralBufferToken{#4}
}


% 3. Relativ
%
% #1: Relative to (z.B. below=1 of ...)
% #2: Text/Name
% #3: Options (node)

\def\UMLActivityCentralBufferRelativeTo#1#2#3{
	\node [rectangle,draw,fill=green!15,inner ysep=9pt,inner xsep=25pt,#3,#1] (#2) {#2};
	\UMLActivityCentralBufferToken{#2}
}


% 4. Relativ mit alternativem Namen
%
% #1: Relative to (z.B. below=1 of ...)
% #2: Text
% #3: Name
% #4: Options

\def\UMLActivityCentralBufferRelativeToAlterName#1#2#3#4{
	\node [rectangle,draw,fill=green!15,inner ysep=9pt,inner xsep=25pt,#4,#1] (#3) {#2};
	\UMLActivityCentralBufferToken{#3}
}


% Token

\def\UMLActivityCentralBufferToken#1{
	\node[rectangle,draw,scale=.5,fill=yellow!25] at($(#1.west)+(.5,-.05)$) (cb) {CB};
	\draw[-{>[scale width=.5]}] ($(cb.north west)+(0,.15)$) -| (cb.north);
	\draw[-{>[scale width=.5]}] (cb.east) -- ($(cb.east)+(.15,0)$);
}


% 1. N1
% 2. N2
% 3. Path
\def\UMLActivityControlFlow#1#2#3{
	\draw[blue,->] (#1) #3 (#2);
}

% 1. N1
% 2. N2
% 3. Path
% 4. Text
% 5. Text Pos
% 6. Text Options
\def\UMLActivityControlFlowWithGuard#1#2#3#4#5#6{
	\draw[blue,->] (#1) #3 (#2);
	\node[#6] at ($(#1)!#5!(#2)$) {\tiny{[#4]}};
}

% 1. N1
% 2. N2
% 3. Path
% 4. Text
% 5. Text Pos
% 6. Text Options
\def\UMLActivityControlFlowWithText#1#2#3#4#5#6{
	\draw[blue,->] (#1) #3 (#2);
	\node[#6] at ($(#1)!#5!(#2)$) {\tiny{#4}};
}



% UML Activity Data Flow (various)

\def\UMLActivityDataFlowPort#1#2#3{
	\draw[fill=#3] ($(#1)-(.15,.15)$) rectangle ($(#1)+(.15,.15)$);
	\draw[->,rotate=#2] ($(#1)-(.1,0)$) -- ($(#1)+(.1,0)$);
}

\def\UMLActivityDataFlow#1#2#3#4#5{
	\draw[brown!50!black,-{>[sep=3pt]>}] (#1) #3 (#2);
	\UMLActivityDataFlowPort{#1}{#4}{blue!25}
	\UMLActivityDataFlowPort{#2}{#5}{orange!25}
}

% 1. N1
% 2. N2
% 3. Path
% 4. Text
% 5. Text Options N1
% 6. Text Options N2
% 7. Arrow Direction (Deg) N1
% 8. Arrow Direction (Deg) N2

\def\UMLActivityDataFlowWithText#1#2#3#4#5#6#7#8{
	\draw[brown!50!black,-{>[sep=3pt]>}] (#1) #3 (#2);
	\node[#5] at (#1) {\tiny{#4}};
	\node[#6] at (#2) {\tiny{#4}};
	\UMLActivityDataFlowPort{#1}{#7}{blue!25}
	\UMLActivityDataFlowPort{#2}{#8}{orange!25}
}

\def\UMLActivityDataFlowNoPorts#1#2#3{
	\draw[brown!50!black,->] (#1) #3 (#2);
}

\def\UMLActivityDataFlowIn#1#2#3#4{
	\draw[brown!50!black,-{>[sep=3pt]>}] (#1) #3 (#2);
	\UMLActivityDataFlowPort{#2}{#4}{orange!25}
}

% 1. N1
% 2. N2
% 3. Path
% 4. Text
% 5. Text Pos
% 6. Text Options
% 7. Arrow Direction (Deg)

\def\UMLActivityDataFlowInWithText#1#2#3#4#5#6#7{
	\draw[brown!50!black,-{>[sep=3pt]>}] (#1) #3 (#2);
	\node[#6] at ($(#1)!#5!(#2)$) {\tiny{#4}};
	\UMLActivityDataFlowPort{#2}{#7}{orange!25}
}

\def\UMLActivityDataFlowOut#1#2#3#4{
	\draw[brown!50!black,->] (#1) #3 (#2);
	\UMLActivityDataFlowPort{#1}{#4}{blue!25}
}

\def\UMLActivityDataFlowOutWithText#1#2#3#4#5#6#7{
	\draw[brown!50!black,->] (#1) #3 (#2);
	\node[#6] at ($(#1)!#5!(#2)$) {\tiny{#4}};
	\UMLActivityDataFlowPort{#1}{#7}{blue!25}
}


% Nodes

\def\UMLActivityInitialNode#1#2{
	\node[circle,draw,inner sep=5pt,fill=black] at (#1,#2) (initial) {};
}

\def\UMLActivityInitialNodeRelativeTo#1{
	\node[circle,draw,inner sep=5pt,fill=black,#1] (initial) {};
}

\def\UMLActivityFinalNode#1#2#3{
	\draw[fill=black] (#1,#2) circle (.2cm);
	\node[circle,draw,inner sep=5pt] at (#1,#2) (#3) {};
}

\def\UMLActivityFinalNodeRelativeTo#1#2{
	\node[circle,draw,inner sep=5pt,#1] (#2) {};
	\node[circle,draw,inner sep=4pt,fill=black] at (#2) {};
}

\def\UMLActivityExitNode#1#2#3{
	\node[circle,draw,fill=white,inner sep=5pt] at (#1,#2) (#3) {};
	\draw (#3.north west) -- (#3.south east);
	\draw (#3.north east) -- (#3.south west);
}

\def\UMLActivityExitNodeRelativeTo#1#2{
	\node[circle,draw,fill=white,inner sep=5pt,#1] (#2) {};
	\draw (#2.north west) -- (#2.south east);
	\draw (#2.north east) -- (#2.south west);
}

\def\UMLActivityDescisionNode#1#2#3{
	\node[diamond,draw,fill=black!10,aspect=.75] at (#1,#2) (#3) {};
}

\def\UMLActivityDescisionNodeRelativeTo#1#2{
	\node[diamond,draw,fill=black!10,aspect=.75,#1] (#2) {};
}

% 1. Relative to
% 2. Name
% 3. Text
% 4. Text Options
\def\UMLActivityWaitTimeActionRelativeTo#1#2#3#4#5{
	\node[#1,minimum size=.5cm] (#2) {};
	\node[#5] at (#4) {\tiny{#3}};
	\draw ($(#2) - (.25,.25)$) -- ($(#2) + (.25,.25)$) -- ($(#2) + (-.25,.25)$) -- ($(#2) + (.25,-.25)$) -- cycle;
	%\node[isosceles triangle,draw,#1,rotate=90,below left] (#2_bottom) {};
}

% 1. X
% 2. Y
% 3. Name
% 4. Size
\def\UMLActivityConcurrentNodeH#1#2#3#4{
	\node[rectangle,draw,fill=black,inner ysep=1pt,inner xsep=#4/2] at (#1,#2) (#3) {};
}

% 1. Relative to
% 2. Name
% 3. Size
\def\UMLActivityConcurrentNodeHRelativeTo#1#2#3{
	\node[rectangle,draw,fill=black,inner ysep=1pt,inner xsep=#3/2,#1] (#2) {};
}

% 1. X
% 2. Y
% 3. Name
% 4. Size
\def\UMLActivityConcurrentNodeV#1#2#3#4{
	\node[rectangle,draw,fill=black,inner xsep=1pt,inner ysep=#4/2] at (#1,#2) (#3) {};
}

% 1. Relative to
% 2. Name
% 3. Size
% 4. Options
\def\UMLActivityConcurrentNodeVRelativeTo#1#2#3#4{
	\node[rectangle,draw,fill=black,inner xsep=1pt,inner ysep=#3/2,#1,#4] (#2) {};
}

\def\UMLActivitySwimlane#1#2#3#4{
	\node[black!25,rectangle split, rectangle split parts = 2,rectangle split empty part height=#3,draw,fill=yellow!10,inner ysep=7pt,inner xsep=15pt,minimum size=#2,#4] (#1) {#1 \nodepart{second}};
	\UMLComponentToken{#1}{yellow!10}{black!25}
}

\def\UMLActivityExternSwimlane#1#2#3#4{
	\node [black!25,rectangle split, rectangle split parts = 2,draw,fill=black!5,inner ysep=7pt,inner xsep=15pt,rectangle split empty part height=#3,minimum size=#2,#4] (#1) {#1 \nodepart{second}};
	\UMLComponentToken{#1}{black!5}{black!25}
}

%%%%%%%%%%%%%%%%%%%%%%%%%%%%%%%%%%%%%%%%%%%%%%%%%%%%%%%%%%%%%%%%%%%%%%%%%%
% State Diagram
%%%%%%%%%%%%%%%%%%%%%%%%%%%%%%%%%%%%%%%%%%%%%%%%%%%%%%%%%%%%%%%%%%%%%%%%%
%
% UML State Diagramm Object State (various)
%
% 1. Normal                         -> \UMLActivityState{ X }{ Y }{ Text/Name }{ Options (node) }
% 2. Normal mit alternativem Namen  -> \UMLActivityStateAlterName{ X }{ Y }{ Text }{ Name }{ Options (node) }
% 3. Relativ                        -> \UMLActivityStateRelativeTo{ Relative to }{ Text/Name }{ Options (node) }
% 4. Relativ mit alternativem Namen -> \UMLActivityStateRelativeToAlterName{ Relative to }{ Text }{ Name }{ Options (node) }
%
%

% 1. Normal
%
% #1: X
% #2: Y
% #3: Text/Name
% #4: Options (node)

\def\UMLStateObjectState#1#2#3#4{
	\node [rectangle,rounded corners,draw,fill=black!10,inner ysep=9pt,inner xsep=10pt,#4] at(#1,#2) (#3) {#3};
}

% 2. Alternativer Name
%
% #1: X
% #2: Y
% #3: Text
% #4: Name
% #5: Options (node)

\def\UMLStateObjectStateAlterName#1#2#3#4#5{
	\node [rectangle,rounded corners,draw,fill=black!10,inner ysep=9pt,inner xsep=10pt,#5] at(#1,#2) (#4) {#3};
}


% 3. Relativ
%
% #1: Relative to (z.B. below=1 of ...)
% #2: Text/Name
% #3: Options (node)

\def\UMLStateObjectStateRelativeTo#1#2#3{
	\node [rectangle,rounded corners,draw,fill=black!10,inner ysep=9pt,inner xsep=10pt,#1,#3] (#2) {#2};
}


% 4. Relativ mit alternativem Namen
%
% #1: Relative to (z.B. below=1 of ...)
% #2: Text
% #3: Name
% #4: Options

\def\UMLStateObjectStateRelativeToAlterName#1#2#3#4{
	\node [rectangle,rounded corners,draw,fill=black!10,inner ysep=9pt,inner xsep=10pt,#1,#4] (#3) {#2};
}

% 1. N1
% 2. N2
% 3. Path
\def\UMLStateControlFlow#1#2#3{
	\draw[->] (#1) #3 (#2);
}

% 1. N1
% 2. N2
% 3. Path
% 4. Guard Text
% 5. Guard Pos
% 6. Guard Options (Node)
\def\UMLStateControlFlowWithGuard#1#2#3#4#5#6{
	\draw[->] (#1) #3 (#2);
	\node[#6] at ($(#1)!#5!(#2)$) {\tiny{[#4]}};
}

% 1. N1
% 2. N2
% 3. Path
% 4. Text Text
% 5. Text Pos
% 6. Text Options (Node)
\def\UMLStateControlFlowWithText#1#2#3#4#5#6{
	\draw[->] (#1) #3 (#2);
	\node[#6] at ($(#1)!#5!(#2)$) {\tiny{#4}};
}

% 1. N1
% 2. N2
% 3. Path
% 4. Guard Text
% 5. Guard Pos
% 6. Guard Options (Node)
% 7. Text Text
% 8. Text Pos
% 9. Text Options (Node)
\def\UMLStateControlFlowWithGuardAndText#1#2#3#4#5#6#7#8#9{
	\draw[->] (#1) #3 (#2);
	\node[#6] at ($(#1)!#5!(#2)$) {\tiny{[#4]}};
	\node[#9] at ($(#1)!#8!(#2)$) {\tiny{#7}};
}


%\input{liberm}

\begin{document}

% Custom Titelseite
\begin{titlepage}
	\begin{center}
		{\Huge{\underline{\textbf{Softwaretechnik II $-$ Praktikum}}}} \\
	\vspace{3cm}
		{\Huge{\textbf{Subsystem 4 $-$ Zubereitung}}} \\
	\vspace{3cm}
		{\Huge{Eine Dokumentation von:}} \\
	\vspace{2cm}
		\huge{J. Fa{\ss}bender} \\
	\vspace{.5cm}
		\huge{J. Gobelet} \\
	\vspace{.5cm}
		\huge{L. Gobelet} \\
	\vspace{.5cm}
		\huge{E. G\"odel}
	\end{center}
\end{titlepage}

\tableofcontents
\newpage
\listoffigures

\section{Meilenstein 1 $-$ Datenzugriffsschicht}

\subsection{Teilaufgabe 1: Ausschnitt aus Logischem DM mit Entities und Value Objects}

\subsubsection{Klassendiagramm}

% hallo {{{

% }}}
\begin{center}
\begin{tikzpicture}

% Ebene 0 {{{
  \UMLClassAlterName
    {0}
    {0}
    {
      \textbf{Gericht}
        \nodepart{second}
      Name: String \\
      Details: String \\
      Preis: double
    }
    {gericht}
    {minimum width=6cm,text width=2.75cm}

  \UMLClassRelativeToAlterName
    {right=3 of gericht}
    {\textbf{Zutat}\nodepart{second}Name: String}
    {zutat}
    {minimum width=6cm}
% }}}

% Ebene -1 {{{
  \UMLClassRelativeToAlterName
    {below=3 of gericht}
    {\textbf{Speise}\nodepart{second}Name: String}
    {speise}
    {minimum width=6cm}
% }}}

% Ebene -2 {{{
  \UMLClassRelativeToAlterName
    {below=3 of speise}
    {
      \textbf{Zubereitungsanleitung}
        \nodepart{second}
      Anleitung: String
    }
    {za}
    {minimum width=6cm,fill=red!25}

  \UMLClassRelativeToAlterName
    {right=3 of za}
    {\textbf{Zutatenangabe}\nodepart{second}Menge: int}
    {zutatenangabe}
    {minimum width=6cm,fill=red!25}
% }}}

% Verbindungen {{{
  % gericht -- speise {{{
  \draw[umlaggreg-] (gericht.south) -- (speise.north);
  \node[below left=.25 of gericht.south] {*};
  \node[above left=.25 of speise.north] {*};
  \node[left] at ($(gericht)!.5!(speise)$) {Teil von};
  % }}}
  % speise -- za {{{
  \draw[umlcompo-] (speise.south) -- (za.north);
  \node[below left=.25 of speise.south] {1};
  \node[above left=.25 of za.north] {1};
  \node[left] at ($(speise)!.5!(za)$) {beschreibt};
  % }}}
  % zutatenangabe -- za {{{
  \draw[-umlcompo] (zutatenangabe.west) -- (za.east);
  \node[above left=.25 of zutatenangabe.west] {1..*};
  \node[above right=.25 of za.east] {1};
  \node[below] at ($(zutatenangabe)!.5!(za)$) {ben\"otigt};
  % }}}
  % zutatenangabe -- zutat {{{
  \draw (zutatenangabe.north) -- (zutat.south);
  \node[above left=.25 of zutatenangabe.north] {*};
  \node[below left=.25 of zutat.south] {1};
  \node[left] at ($(zutatenangabe)!.5!(zutat)$) {ben\"otigt};
  % }}}
% }}}
\end{tikzpicture}
\end{center}


\subsubsection{fachliches Glossar}


\section{Meilenstein 2 $-$ Komponentenschnitt}

% Teilaufgabe 1 {{{
\subsection{Teilaufgabe 1: Vorbereitung des Komponentenschnitts}

% Liste der Geschaeftsobjekte {{{
\subsubsection{Liste der Gesch\"aftsobjekte}

\begin{itemize}

  \item Arbeitsplatz

  \item Bestellung

  \item Gericht

  \item Sitzplatz

  \item Speisekarte

  \item Zubereitungsanleitung

  \item Zutat

  \item Zutatenposition

\end{itemize}
% }}}

% Liste der Use Cases {{{
\subsubsection{Liste der Use Cases}

\begin{itemize}

  \item Am Arbeitsplatz an-/abmelden

  \item Gericht bestellen

  \item Gericht zubereiten

\end{itemize}
% }}}

% Liste der Umsysteme {{{
\subsubsection{Liste der Umsysteme}

\begin{tabu} to \linewidth {X|X|X}
% Headerzeile {{{
\hline
\rowcolor{codebordercolor}
Umsystem &Was geschieht zwischen Umsystem und unserem Subsystem?
  &Schnittstelle angeboten oder aufgerufen \\
% }}}
% Rezeptverwaltung {{{
\hline
Rezeptverwaltung &Rezeptverwaltung verwaltet die Gesch\"aftsobjekte Gericht,
  Zubereitungsanleitung und Speisekarte. Der Gast fragt \"uber das ihm zur
  Verf\"ugung gestellte Frontend die Speisekarte und die Gerichte ab, w\"ahrend
  der Koch an seinem Terminal die Zubereitungsanleitung und die hiermit verbundenen
  Zutatenpositionen, angezeigt bekommt.

  &Aufruf einer Schnittstelle zur Rezeptverwaltung \\
% }}}
% Lagerverwaltung {{{
\hline
Lagerverwaltung &Abfrage zum Zutatenbestand &Aufruf einer Schnittstelle zur
  Lagerverwaltung \\
% }}}
% Lagerverwaltung {{{
\hline
Lagerverwaltung &Angabe zur Zutantenentnahme (kann auch \"uber die gleiche
  Schnittstelle, die im obrigen Tabelleneintrag spezifiziert ist, realisiert
  werden)

  &Aufruf einer Schnittstelle zur Lagerverwaltung \\
% }}}
% Buchhaltung {{{
\hline
Buchhaltung &Abfrage der Bestellungen &Schnittstelle wird Buchhaltung zur verf\"ugung
  gestellt \\
% }}}
\hline
\end{tabu}

% Erlaeuterung {{{
\textbf{Erl\"auterung} \\

Wir legen redundant zur Lagerverwaltung unsere eingene Verwaltung
mit Angaben zum Zutatenbestand an, um auch bei Nichterreichbarkeit
der Lagerverwaltung funktionsf\"ahig zu bleiben, da unser Subsystem
essentiell f\"ur den Umsatz verantwortlich ist und ein Ausfall, das
hei{\ss}t in diesem Fall der Zustand, dass eine Zutat nicht mehr in
ben\"otigter Menge im Lager zur Verf\"ugung steht, nicht auf Grund
technischer Probleme eintreten sollte.

Allerdings stellen wir keinen Anspruch auf absolute Richtigkeit unserer
Zutantenbestandsverwaltung, da wir nur die Ereignisse unseres Subsystems,
das hei{\ss}t in diesem Fall die Entnahme einer Zutat zur Zubereitung,
protokollieren und die restlichen Angaben aus der Lagerverwaltung stammen.

Ist diese nun nicht erreichbar, verwendet unsere Zutatenbestandsverwaltung
mitunter veraltete Daten, was wir nicht mit einbeziehen.

Der Lagerverwaltung wird die Entnahme von unserem Subsystem aus mitgeteilt.

F\"ur den kompletten Synchronisationsprozess zwischen den beiden Systemen
stellt uns die Lagerverwaltung zwei Schnittstellen (oder eine, die beide
Aufgaben - Entnahme mitteilen und Zutatenbestand abfragen - zusammenfasst)
zur Verf\"ugung.

Zus\"atzlich haben wir eine Schnittstelle f\"ur die Buchhaltung angelegt.
Diese ist zwar kein explizites Subsystem, wird aber, unserer Meinung nach,
im Betriebsumfeld h\"ochstwahrscheinlich als eigenes Subsystem existieren
und unsere Schnittstelle zu den Bestellungen (im Endeffekt der Unternehmens\-
umsatz aus dem Hauptgesch\"aft) nutzen wollen.
% }}}

% }}}

% }}}

% Teilaufgabe 2 {{{
\subsection{Teilaufgabe 2: Ermittlung der verschiedenen Komponenten-Typen}

% Schritt 1 {{{
\subsubsection{Schritt 1: Gesch\"aftsobjekte in zusammenh\"angende Gruppen einteilen}

\begin{tabu} to \linewidth {X|X|X}
% Headerzeile {{{
\hline
\rowcolor{codebordercolor}
Datenkomponente &Zugeordnete Gesch\"aftsobjekte &Erkl\"arung \\
% }}}
% Bestellungskomponente {{{
\hline
Bestelldaten &Bestellung &Die einzigen Daten die in diesem Subsystem tas\"achlich generiert
  werden. Da die Bestellungen sehr wichtig f\"ur das Hauptgesch\"aft der Firma ist, es das einzige
  Datenobjekt mit Implementierung eines Create-Interfaces (Factory) ist und auch sonst nicht in unsere
  sonstigen Datenkomponenten passt, wird die Bestellung, unserer Meinung nach, in einer eigenen
  Komponente implementiert.  \\
% }}}
% Standortkomponente {{{
\hline
Standortdaten &Arbeitsplatz, Sitzplatz &Diese Daten \"andern sich \"au{\ss}erst selten (und auch nicht
  in unserem Subsystem) und umfassen im Vergleich zu anderen Komponenten wenig Datens\"atze und k\"onnen
  deshalb, unserer Meinung nach, gut zusammengefasst werden.\\
% }}}
% Gerichtskomponente {{{
\hline
Gerichtsdaten &Gericht, Speisekarte, Zubereitungsanleitung, Zutat, Zutatenposition
  &Stammdaten die f\"ur unseren Prozess der Zubereitung essentiell sind. Diese Daten
  stammen nicht aus unserem Subsystem, sondern sind \"uber Schnittstellen abrufbar,
  sowohl von der Lagerverwaltung (Zutat), als auch von der Rezepteverwaltung (Gericht,
  Speisekarte, Zubereitungsanleitung, Zutatenposition). Unsere Datenkomponente greift
  \"uber Adapterkomponenten auf diese Schnittstellen zu.\\
% }}}
\hline
\end{tabu}

% }}}

% Schritt 2 {{{
\subsubsection{Schritt 2: Use Cases auf Daten/Logik analysieren}

\begin{tabu} to \linewidth {X|X|X}
% Headerzeile {{{
\hline
\rowcolor{codebordercolor}
Daten-/Logikkomponente &Zugeordnete(r) Use Case(s) &Erkl\"arung \\
% }}}
% Bestellabwicklung {{{
\hline
Bestellabwicklung (Logik) &Am Arbeitsplatz an-/abmelden, Gericht bestellen, Gericht zubereiten
  &Unser "`Backend"', was ab der Bestellungsaufgabe den Zubereitungsprozess steuert. Die
  Komponente umfasst die Vergabewarteschlange mit den besetzten und freien Arbeitspl\"atzen
  und \"ubernimmt die Zuweisung, sobald eine Bestellung von einem Clienten eingeht. Sobald ein
  Gericht fertig zubereitet ist und der Koch dies seinem Terminal mitteilt, \"ubernimmt diese
  Komponente auch die Anzeige der Ordernummer (im Gast-UI). Da dies alles vom Umfang her eher
  kleinere Aufgaben sind, haben wir uns dazu entschieden, diese Aufgaben in einer Komponente
  zusammenzufassen.\\
% }}}
\hline
\end{tabu}

% }}}

% Schritt 3 {{{
\subsubsection{Schritt 3: Use Cases auf Nutzer-Interaktion analysieren}

\begin{tabu} to \linewidth {X|X|X|X}
% Headerzeile {{{
\hline
\rowcolor{codebordercolor}
Dialogkomponente &Zugeordnete(r) Use Case(s) &Eigene Fassadenkomponente sinnvoll?
  &Erkl\"arung \\
% }}}
% Zubereitungs-UI {{{
\hline
Zubereitungs-UI &Gericht zubereiten &Ja &Fassadenkomponente zur
  Orchestrierung der Gerichtszubereitung. \\
% }}}
% An-/Abmeldungs-UI {{{
\hline
An-/Abmeldungs-UI &Am Arbeitsplatz an-/abmelden &Ja &Fassadenkomponente f\"ur den Zugriff auf
  Datenkomponente "`Standortdaten"' (Read- und Updateoperationen auf den Arbeitsplatz) und
  um das "`Strict Layering"' einzuhalten.\\
% }}}
% Gast-UI {{{
\hline
Gast-UI &Gericht bestellen &Ja &Fassadenkomponente zur Orchestrierung des Bestellvorgangs. \\
% }}}
\hline
\end{tabu}

% }}}

% Schritt 4 {{{
\subsubsection{Schritt 4: Angebot von externen Schnittstellen}

\begin{tabu} to \linewidth {X|X|X}
% Headerzeile {{{
\hline
\rowcolor{codebordercolor}
Umsystem/Schnittstelle &Eigene Fassadenkomponente sinnvoll? &Erkl\"arung \\
% }}}
% Buchhaltung {{{
\hline
Buchhaltung &Ja &Da die Buchhaltung lesenden Zugriff auf
  usere Bestellungen haben soll, ist es notwendig eine spezialisierte Komponente
  hierf\"ur anzulegen und nicht, wie intern in unserem Subsystem, den Zugriff
  \"uber die Bestelldatenkomponente zu regeln.\\
% }}}
% Lagerverwaltung {{{
\hline
Lagerverwaltung &Nein &Zugriff erfolgt nur aus der Gerichtsdatenkomponente \"uber die
  Adapterkomponente der Lagerverwaltung, weshalb, unserer Meinung nach, keine Fassaden\-
  komponente notwendig ist. \\
% }}}
% Rezeptverwaltung {{{
\hline
Rezeptverwaltung &Nein &Zugriff erfolgt nur aus der Gerichtsdatenkomponente \"uber die
  Adapterkomponente der Rezepteverwaltung, weshalb, unserer Meinung nach, keine Fassaden\-
  komponente notwendig ist. \\
% }}}
\hline
\end{tabu}

% }}}

% Schritt 5 {{{
\subsubsection{Schritt 5: Aufruf von externen Schnittstellen/Umsystemen}

\begin{tabu} to \linewidth {X|X|X}
% Headerzeile {{{
\hline
\rowcolor{codebordercolor}
Umsystem/Schnittstelle &Adapterkomponente sinnvoll? &Erkl\"arung \\
% }}}
% Buchhaltung {{{
\hline
Buchhaltung &Nein &Bereits spezialisierte Fassadenkomponente vorhanden. \\
% }}}
% Lagerverwaltung {{{
\hline
Lagerverwaltung &Ja &Adapterkomponente f\"ur unsere Gerichtsdatenkomponente,
  die die Lese- und Schreibvorg\"ange zur Verf\"ugung stellt und gleichzeitig
  bei Ausf\"allen als "`Anti-Corruption-Layer"' fungiert. \\
% }}}
% Rezeptverwaltung {{{
\hline
Rezeptverwaltung &Ja &Adapterkomponente f\"ur unsere Gerichtsdatenkomponente,
  die die Lesevorg\"ange zur Verf\"ugung stellt und gleichzeitig bei Ausf\"allen
  als "`Anti-Corruption-Layer"' fungiert. \\
% }}}
\hline
\end{tabu}

% }}}

% }}}

% Teilaufgabe 3 {{{
\subsection{Teilaufgabe 3: Komponentendiagramm}
\begin{figure}[H]
\begin{center}
\begin{tikzpicture}
  % Column 1 {{{

  % Row 1 {{{
  \UMLSComponent
    {0}
    {0}
    {Gast-UI}
    {fill=green!25,text width=3cm}
  % }}}

  % Row 2 {{{
  \UMLSComponentRelativeToAlterName
    {below=3 of Gast-UI}
    {Bestellvorgangs\-fassade}
    {bo}
    {fill=purple,text width=3cm}
  % }}}

  % Row 3 {{{
  \UMLSComponentRelativeTo
    {below=7.5 of bo}
    {Standortdaten}
    {fill=blue!25,text width=3cm}
  % }}}

  % Row 4 {{{
  \UMLSComponentRelativeToAlterName
    {below=2 of Standortdaten}
    {Buchhaltungs-API}
    {ba}
    {fill=purple,text width=3cm}
  % }}}

  % Row 5 {{{
  \UMLSComponentRelativeToAlterName
    {below=1.5 of ba}
    {$<<$Buchhal\-tung$>>$}
    {bh}
    {fill=black!50,text width=3cm}
  % }}}

  % }}}

  % Column 2 {{{

  % Row 1 {{{
  \UMLSComponentRelativeTo
    {right=3 of Gast-UI}
    {Bestellabwicklung}
    {fill=orange,text width=3cm}
  \UMLSComponentRelativeToAlterName
    {right=3 of Gast-UI}
    {An-/Abmel\-dungs-UI}
    {aaui}
    {fill=green!25,text width=3cm}
  % }}}

  % Row 2 {{{
  \UMLSComponentRelativeToAlterName
    {right=3 of bo}
    {An-/Abmel\-dungs\-fassade}
    {aao}
    {fill=purple,text width=3cm}
  % }}}

  % Row 2.5 {{{
  \UMLSComponentRelativeTo
    {below=3 of aao}
    {Bestellabwicklung}
    {fill=orange,text width=3cm}
  % }}}

  % Row 3 {{{
  \UMLSComponentRelativeTo
    {right=3 of Standortdaten}
    {Bestelldaten}
    {fill=blue!25,text width=3cm}
  % }}}

  % Row 4 {{{
  \UMLSComponentRelativeTo
    {right=3 of ba}
    {Rezeptadapter}
    {fill=black!25,text width=3cm}
  % }}}

  % Row 5 {{{
  \UMLSComponentRelativeToAlterName
    {right=3 of bh}
    {$<<$Rezeptver\-waltung$>>$}
    {rvw}
    {fill=black!50,text width=3cm}
  % }}}

  % }}}

  % Column 3 {{{

  % Row 1 {{{
  \UMLSComponentRelativeTo
    {right = 3 of aaui}
    {Zubereitungs-UI}
    {fill=green!25, text width=3cm}
  % }}}

  % Row 2 {{{
  \UMLSComponentRelativeToAlterName
    {below=3 of Zubereitungs-UI}
    {Zubereitungs\-fassade}
    {zo}
    {fill=purple,text width=3cm}
  % }}}

  % Row 3 {{{
  \UMLSComponentRelativeTo
    {right=3 of Bestelldaten}
    {Gerichtsdaten}
    {fill=blue!25,text width=3cm}
  % }}}

  % Row 4 {{{
  \UMLSComponentRelativeToAlterName
    {right=3 of Rezeptadapter}
    {Lagerverwal\-tungsadapter}
    {lvwa}
    {fill=black!25,text width=3cm}
  % }}}

  % Row 5 {{{
  \UMLSComponentRelativeToAlterName
    {right=3 of rvw}
    {$<<$Lagerver\-waltung$>>$}
    {lvw}
    {fill=black!50,text width=3cm}
  % }}}

  % }}}

  % Connections {{{

    % Gast-UI -> bo {{{
    \draw [-{>[sep=3.5]>}] (Gast-UI.south) -- (bo.north);

      % Portnumber {{{
      \node[above right=.5 of bo.north] {1}
        edge (bo.north);
      % }}}

    \UMLComponentPort
      {Gast-UI.south}
      {fill=white}

    \UMLComponentPort
      {bo.north}
      {fill=white}

    % }}}

    % aaui -> aao {{{
    \draw [-{>[sep=3.5]>}] (aaui.south) -- (aao.north);

      % Portnumber {{{
      \node[above right=.5 of aao.north] {2}
        edge (aao.north);
      % }}}

    \UMLComponentPort
      {aaui.south}
      {fill=white}

    \UMLComponentPort
      {aao.north}
      {fill=white}
    % }}}

    % bo -> Bestellabwicklung {{{
    \draw [-{>[sep=3.5]>}]
      (bo.east)
      -|
      ($(bo)!.5!(Bestellabwicklung)$)
      |-
      (Bestellabwicklung.west);

      % Portnumber {{{
      \node[above left=.5 of Bestellabwicklung.west] {4}
        edge (Bestellabwicklung.west);
      % }}}

    \UMLComponentPort
      {bo.east}
      {fill=white}

    \UMLComponentPort
      {Bestellabwicklung.west}
      {fill=white}
    % }}}

    % aao -> Bestellabwicklung {{{
    \draw [-{>[sep=3.5]>}] (aao.south) -- (Bestellabwicklung.north);

      % Portnumber {{{
      \node[above right=.5 of Bestellabwicklung.north] {5}
        edge (Bestellabwicklung.north);
      % }}}

    \UMLComponentPort
      {aao.south}
      {fill=white}

    \UMLComponentPort
      {Bestellabwicklung.north}
      {fill=white}
    % }}}

    % Bestellabwicklung -> zo {{{
    \draw [-{>[sep=3.5]>}]
      (Bestellabwicklung.east)
      -|
      ($(Bestellabwicklung)!.5!(zo)$)
      |-
      (zo.west);

      % Portnumber {{{
      \node[below left=.5 of zo.west] {6}
        edge (zo.west);
      % }}}

    \UMLComponentPort
      {Bestellabwicklung.east}
      {fill=white}

    \UMLComponentPort
      {zo.west}
      {fill=white}
    % }}}

    % Zubereitungs-UI -> zo {{{
    \draw [-{>[sep=3.5]>}] (Zubereitungs-UI.south) -- (zo.north);

      % Portnumber {{{
      \node[above left=.5 of zo.north] {3}
        edge (zo.north);
      % }}}

    \UMLComponentPort
      {Zubereitungs-UI.south}
      {fill=white}

    \UMLComponentPort
      {zo.north}
      {fill=white}
    % }}}

    % bo -> Bestelldaten {{{
    \draw [-{>[sep=3.5]>}]
      (bo.south)
      |-
      ($(bo)!.75!(Bestelldaten)$)
      -|
      (Bestelldaten.158);

      % Portnumber {{{
      \node[above right=.5 of Bestelldaten.158] {8}
        edge (Bestelldaten.158);
      % }}}

    \UMLComponentPort
      {bo.south}
      {fill=white}

    \UMLComponentPort
      {Bestelldaten.158}
      {fill=white}
    % }}}

    % bo -> Standortdaten {{{
    \draw [-{>[sep=3.5]>}] (bo.210) -- (Standortdaten.158);

      % Portnumber {{{
      \node[above right=.5 of Standortdaten.158] {6}
        edge (Standortdaten.158);
      % }}}

    \UMLComponentPort
      {bo.210}
      {fill=white}

    \UMLComponentPort
      {Standortdaten.158}
      {fill=white}
    % }}}

    % bo -> Gerichtsdaten {{{
    \draw [-{>[sep=3.5]>}]
      (bo.330)
      |-
      ($(bo)!.65!(Gerichtsdaten)$)
      -|
      (Gerichtsdaten.158);

      % Portnumber {{{
      \node[above right=.5 of Gerichtsdaten.158] {10}
        edge (Gerichtsdaten.158);
      % }}}

    \UMLComponentPort
      {bo.330}
      {fill=white}

    \UMLComponentPort
      {Gerichtsdaten.158}
      {fill=white}
    % }}}

    % aao -> Standortdaten {{{
    \draw [-{>[sep=3.5]>}]
      (aao.210)
      |-
      ($(aao)!.35!(Standortdaten)$)
      -|
      (Standortdaten.46);

      % Portnumber {{{
      \node[above right=.5 of Standortdaten.46] {7}
        edge (Standortdaten.46);
      % }}}

    \UMLComponentPort
      {aao.210}
      {fill=white}

    \UMLComponentPort
      {Standortdaten.46}
      {fill=white}
    % }}}

    % zo -> Gerichtsdaten {{{
    \draw [-{>[sep=3.5]>}] (zo.330) -- (Gerichtsdaten.22);

      % Portnumber {{{
      \node[above left=.5 of Gerichtsdaten.22] {11}
        edge (Gerichtsdaten.22);
      % }}}

    \UMLComponentPort
      {zo.330}
      {fill=white}

    \UMLComponentPort
      {Gerichtsdaten.22}
      {fill=white}
    % }}}

    % zo -> Bestelldaten {{{
    \draw [-{>[sep=3.5]>}]
      (zo.south)
      |-
      ($(zo)!.75!(Bestelldaten)$)
      -|
      (Bestelldaten.22);

      % Portnumber {{{
      \node[above left=.5 of Bestelldaten.22] {9}
        edge (Bestelldaten.22);
      % }}}

    \UMLComponentPort
      {zo.south}
      {fill=white}

    \UMLComponentPort
      {Bestelldaten.22}
      {fill=white}
    % }}}

    % ba -> Bestelldaten {{{
    \draw [-{>[sep=3.5]>}]
      (ba.north)
      |-
      ($(ba)!.7!(Bestelldaten)$)
      -|
      (Bestelldaten.south);

      % Portnumber {{{
      \node[below right=.5 of Bestelldaten.south] {12}
        edge (Bestelldaten.south);
      % }}}

    \UMLComponentPort
      {ba.north}
      {fill=white}

    \UMLComponentPort
      {Bestelldaten.south}
      {fill=white}
    % }}}

    % Gerichtsdaten -> Rezeptadapter {{{
    \draw [-{>[sep=3.5]>}]
      (Gerichtsdaten.230)
      |-
      ($(Gerichtsdaten)!.5!(Rezeptadapter)$)
      -|
      (Rezeptadapter.north);

      % Portnumber {{{
      \node[above right=.5 of Rezeptadapter.north] {13}
        edge (Rezeptadapter.north);
      % }}}

    \UMLComponentPort
      {Gerichtsdaten.230}
      {fill=white}

    \UMLComponentPort
      {Rezeptadapter.north}
      {fill=white}
    % }}}

    % Gerichtsdaten -> lvwa {{{
    \draw [-{>[sep=3.5]>}] (Gerichtsdaten.310) -- (lvwa.60);

      % Portnumber {{{
      \node[above left=.5 of lvwa.60] {14}
        edge (lvwa.60);
      % }}}

    \UMLComponentPort
      {Gerichtsdaten.310}
      {fill=white}

    \UMLComponentPort
      {lvwa.60}
      {fill=white}
    % }}}

    % ba -> bh {{{
    \draw [-umlportprovider] (ba)--($(ba)!.5!(bh)$);
    \draw [-umlportcaller]   (bh)--($(bh)!.5!(ba)$);

    \UMLComponentPort
      {ba.south}
      {fill=white}

    \UMLComponentPort
      {bh.north}
      {fill=white}
    % }}}

    % Rezeptadapter -> rvw {{{
    \draw [-umlportcaller]   (Rezeptadapter)--($(Rezeptadapter.south)!.5!(rvw.north)$);
    \draw [-umlportprovider] (rvw)--($(rvw.north)!.5!(Rezeptadapter.south)$);

    \UMLComponentPort
      {Rezeptadapter.south}
      {fill=white}

    \UMLComponentPort
      {rvw.north}
      {fill=white}
    % }}}

    % lvwa -> lvw {{{
    \draw [-umlportcaller]   (lvwa)--($(lvwa)!.5!(lvw)$);
    \draw [-umlportprovider] (lvw)--($(lvw)!.5!(lvwa)$);

    \UMLComponentPort
      {lvwa.south}
      {fill=white}

    \UMLComponentPort
      {lvw.north}
      {fill=white}
    % }}}

  % }}}

\end{tikzpicture}
\caption{Komponentendiagramm}
\end{center}
\end{figure}

\newpage

% Legende {{{
\begin{itemize}[label={}]
  \item \tikz[baseline={-3pt}]{
    \node[fill=green!25,green!25,inner ysep=3pt]{X};
  }: Dialogkomponente

  \item \tikz[baseline={-3pt}]{
    \node[fill=purple,purple,inner ysep=3pt]{X};
  }: Fassadenkomponente

  \item \tikz[baseline={-3pt}]{
    \node[fill=blue!25,blue!25,inner ysep=3pt]{X};
  }: Datenkomponente

  \item \tikz[baseline={-3pt}]{
    \node[fill=orange,orange,inner ysep=3pt]{X};
  }: Logikkomponente

  \item \tikz[baseline={-3pt}]{
    \node[fill=black!25,black!25,inner ysep=3pt]{X};
  }: Adapterkomponente

  \item \tikz[baseline={-3pt}]{
    \node[fill=black!50,black!50,inner ysep=3pt]{X};
  }: Umsystem
\end{itemize}
% }}}

% }}}


\section{Meilenstein 3 $-$ Spezifikation, Implementierung
  und Demo eines REST-API}

% Teilaufgabe 1 {{{
\subsection{Teilaufgabe 1: Festlegen von Aggregates}

\begin{figure}[H]
\begin{center}
\begin{tikzpicture}

% Ebene 0 {{{
  \UMLClassAlterName
    {0}
    {0}
    {
      \textbf{Gericht}
        \nodepart{second}
      Name: String \\
      Details: String \\
      Preis: double
    }
    {gericht}
    {minimum width=6cm,text width=2.75cm,dashed}

  \UMLClassRelativeToAlterName
    {right=3 of gericht}
    {\textbf{Zutat}\nodepart{second}Name: String}
    {zutat}
    {minimum width=6cm}
% }}}

% Ebene -1 {{{
  \UMLClassRelativeToAlterName
    {below=3 of gericht}
    {\textbf{Speise}\nodepart{second}Name: String}
    {speise}
    {minimum width=6cm}
% }}}

% Ebene -2 {{{
  \UMLClassRelativeToAlterName
    {below=3 of speise}
    {
      \textbf{Zubereitungsanleitung}
        \nodepart{second}
      Anleitung: String
    }
    {za}
    {minimum width=6cm,fill=red!25}

  \UMLClassRelativeToAlterName
    {right=3 of za}
    {\textbf{Zutatenposition}\nodepart{second}Menge: int}
    {zutatenmenge}
    {minimum width=6cm,fill=red!25}
% }}}

% Verbindungen {{{
  % gericht -- speise {{{
  \draw[umlaggreg-] (gericht.south) -- (speise.north);
  \node[below left=.25 of gericht.south] {*};
  \node[above left=.25 of speise.north] {*};
  \node[left] at ($(gericht)!.5!(speise)$) {Teil von};
  % }}}
  % speise -- za {{{
  \draw[umlcompo-] (speise.south) -- (za.north);
  \node[below left=.25 of speise.south] {1};
  \node[above left=.25 of za.north] {1};
  \node[left] at ($(speise)!.5!(za)$) {beschreibt};
  % }}}
  % zutatenmenge -- za {{{
  \draw[-umlcompo] (zutatenmenge.west) -- (za.east);
  \node[above left=.25 of zutatenmenge.west] {1..*};
  \node[above right=.25 of za.east] {1};
  \node[below] at ($(zutatenmenge)!.5!(za)$) {ben\"otigt};
  % }}}
  % zutatenmenge -- zutat {{{
  \draw (zutatenmenge.north) -- (zutat.south);
  \node[above left=.25 of zutatenmenge.north] {*};
  \node[below left=.25 of zutat.south] {1};
  \node[left] at ($(zutatenmenge)!.5!(zutat)$) {ben\"otigt};
  % }}}
% }}}

% Aggregates {{{
\draw[rounded corners, red!75]
  ($(gericht.north) + (0,.5)$)
  --
  ($(gericht.north west) + (-.5,.5)$)
  |-
  ($(zutatenmenge.south east)+(.5,-.5)$)
  |-
  ($(gericht.north west) + (0,.5)$)
  --
  cycle
;
% }}}

\end{tikzpicture}
\end{center}
% Legende {{{
\tikz[baseline={-3pt}]{
  \node[fill=red!25,red!25,inner ysep=3pt]{X};
}: Value Object \\
\tikz[baseline={-3pt}]{
  \node[fill=yellow!25,yellow!25,inner ysep=2pt]{X};
}: Entity \\
\tikz[baseline={-3pt}]{
  \node[draw,red!75,inner ysep=3pt]{\textcolor{white}{X}};
}: Aggregate \\
\tikz[baseline={-3pt}]{
  \node[draw,dashed,inner ysep=3pt]{\textcolor{white}{X}};
}: Aggregate Root
% }}}
\caption{Aggregates}
\end{figure}


Wir sind der Meinung, dass sich die Datenobjekte Gericht,
Speise, Zubereitungsanleitung und Zutatenposition als ein
Aggregate mit Gericht als Aggregate Root zusammenfassen
lassen, da keine Referenzen auf innere Entities existieren
und ein fachlicher Zusammenhang besteht, da ein Gericht aus
Speisen besteht, Speisen eine Zubereitungsanleitung haben
und diese wiederum Zutatenpositionen, ergibt sich hier
ein enges fachliches Geflecht.

Eine m\"ogliche Invariante w\"are, wenn $Gericht.name$ eine
Kombination von den zugeh\"origen Speisen w\"are. Als
Beispiel hierf\"ur: $Gericht.name$: "`Schnitzel mit Pommes
und Salat"'. Daraus lassen sich die Speisen Schnitzel,
Pommes und Salat ableiten.
% }}}

% Teilaufgabe 2 {{{
\subsection{Teilaufgabe 2: Design des REST-API}

F\"ur unser REST-API verwenden wir folgenden Ausschnitt aus
unserem Klassendiagramm aus Meilenstein 1:

\begin{figure}[H]
\begin{center}
\begin{tikzpicture}

  \UMLClassAlterName
    {0}
    {0}
    {
      \textbf{Gericht}
        \nodepart{second}
      Name: String \\
      Details: String \\
      Preis: double
    }
    {gericht}
    {minimum width=6cm,text width=2.75cm, label=A/B}

  \UMLClassRelativeToAlterName
    {below=3 of gericht}
    {\textbf{Speise}\nodepart{second}Name: String}
    {speise}
    {minimum width=6cm,label={below:C}}

  \draw[umlaggreg-] (gericht.south) -- (speise.north);
  \node[below left=.25 of gericht.south] {*};
  \node[above left=.25 of speise.north] {*};
  \node[left] at ($(gericht)!.5!(speise)$) {Teil von};

\end{tikzpicture}
\end{center}
\caption{Ausschnitt Klassendiagramm f\"ur REST-API}
\end{figure}


\begin{tabu} to \linewidth {l|l|X|X|X}
% Headerzeile {{{
\hline
\rowcolor{codebordercolor}
Szen.-Nr. &URI &HTTP Verb &Request-Body &Ressource und
  Aktion \\
% }}}
% A1, A5, BC1 {{{
\hline
A1, A5, BC1 &/gerichte/\{gericht\_id\} &PUT, DELETE
  &\textbf{Nur bei Put:}
    \{
      name=\dots,
      details=\dots,
      preis=\dots
    \}
  &Neues Gericht anlegen, Gericht
  \"uberschreiben, Gericht l\"oschen. \\
% }}}
% A2, A4 {{{
\hline
A2, A4 &/gerichte?filter[preis]$>$\{wert\} &GET & &Alle
  Gerichte $a$ ausgeben, f\"ur die $a.preis > wert$ gilt.
  \\
% }}}
% A3 {{{
\hline
A3 &/gerichte/\{gericht\_id\}/preis &PUT &\{wert\} &Preis
  von Gericht $a$ mit $a.gericht\_id=\{gericht\_id\}$ auf
  $\{wert\}$ setzen. \\
% }}}
% A6 {{{
\hline
A6 &/gerichte/\{gericht\_id\}/id &GET & &Id von Gericht
  abfragen. 404 wird geworfen, falls Gericht nicht
  vorhanden. \\
% }}}
% BC2 {{{
\hline
BC2 &/speisen/\{speise\_id\} &PUT &\{name=\dots\} &Speise
  anlegen, \"uberschreiben. \\
% }}}
% BC3, BC6 {{{
\hline
BC3, BC6 &/gerichte/\{gericht\_id\}/speisen &PUT, DELETE
  &\{speise\_id=\dots\} &Speise einem Gericht hinzuf\"ugen
  oder l\"oschen.
  \\
% }}}
% BC4, BC7 {{{
\hline
BC4, BC7 &/gerichte &GET & &Alle Gerichte ausgeben. \\
% }}}
% BC5 {{{
\hline
BC5 &/speisen &GET & &Alle Speisen ausgeben. \\
% }}}
\hline
\end{tabu}
% }}}

\subsubsection*{Bemerkung}

Bei Szenario BC3 und BC6 haben wir uns entschieden, dass
die Beziehung zwsichen einer Instanz von Gericht und einer
Instanz von Speise \"uber /gerichte/\{gericht\_id\}/speisen
hinzugef\"ugt oder gel\"oscht weren kann. Man h\"atte dies
auch \"uber /speisen/\{speisen\_id\}/gerichte tun k\"onnen,
was wir jedoch f\"ur un\"ubersichtlicher und nicht so
naheliegend wie unsere Variante gehalten haben.

\newpage

% Teilaufgabe 3 {{{
\subsection{Teilaufgabe 3: Implementierung in Spring Data
  JPA / Web MVC}

% Code-Listing {{{
\subsubsection{Code-Listing}

% GerichtRESTController {{{
\begin{mdframed}[style=codebox]
\textbf{GerichtRESTController}
\lstinputlisting[
  language=Java,
  firstline=34,
  lastline=134
]{%
  ../code/ms3RESTSpringDemo/src/main/java/%
  ms3restspringdemo/services/GerichtRestController%
  .java
}
\end{mdframed}
% }}}

% SpeiseRESTController {{{
\begin{mdframed}[style=codebox]
\textbf{SpeiseRESTController}
\lstinputlisting[
  language=Java,
  firstline=28,
  lastline=43
]
{%
  ../code/ms3RESTSpringDemo/src/main/java/%
  ms3restspringdemo/services/SpeiseRestController%
  .java
}
\end{mdframed}
% }}}

% }}}

% Nachweis der Lauffaehigkeit {{{
\subsubsection{Nachweis der Lauff\"ahigkeit}

% Postman-Collection {{{
\textbf{Postman-Collection}
\lstinputlisting[
  language=Java,
]{%
  ms3/static/st-ms3.%
  postman_collection.json
}
% }}}

% Postman-Test-Run {{{
\textbf{Postman-Test-Run}
\lstinputlisting[
  language=Java,
]{%
  ms3/static/st-ms3.%
  postman_test_run.json
}
% }}}

% }}}

% }}}


\end{document}
