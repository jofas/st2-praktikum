\section{Meilenstein 2 $-$ Komponentenschnitt}

% Teilaufgabe 1 {{{
\subsection{Teilaufgabe 1: Vorbereitung des Komponentenschnitts}

% Liste der Geschaeftsobjekte {{{
\subsubsection{Liste der Gesch\"aftsobjekte}

\begin{itemize}

  \item Arbeitsplatz

  \item Bestellung

  \item Gericht

  \item Sitzplatz

  \item Speisekarte

  \item Zubereitungsanleitung

  \item Zutat

  \item Zutatenangabe

\end{itemize}
% }}}

% Liste der Use Cases {{{
\subsubsection{Liste der Use Cases}

\begin{itemize}

  \item Am Arbeitsplatz an-/abmelden

  \item Gericht bestellen

  \item Gericht zubereiten

\end{itemize}
% }}}

% Liste der Umsysteme {{{
\subsubsection{Liste der Umsysteme}

\begin{tabu} to \linewidth {X|X|X}
% Headerzeile {{{
\hline
\rowcolor{codebordercolor}
Umsystem &Was geschieht zwischen Umsystem und unserem Subsystem?
  &Schnittstelle angeboten oder aufgerufen \\
% }}}
% Rezeptverwaltung {{{
\hline
Rezeptverwaltung &Rezeptverwaltung verwaltet die Gesch\"aftsobjekte Gericht,
  Zubereitungsanleitung und Speisekarte. Der Gast fragt \"uber das ihm zur
  Verf\"ugung gestellte Frontend die Speisekarte und die Gerichte ab, w\"ahrend
  der Koch an seinem Terminal die Zubereitungsanleitung angezeigt bekommt.

  &Aufruf einer Schnittstelle zur Rezeptverwaltung \\
% }}}
% Lagerverwaltung {{{
\hline
Lagerverwaltung &Abfrage zum Zutatenbestand &Aufruf einer Schnittstelle zur
  Lagerverwaltung \\
% }}}
% Lagerverwaltung {{{
\hline
Lagerverwaltung &Angabe zur Zutantenentnahme (kann auch \"uber die gleiche
  Schnittstelle, die im obrigen Tabelleneintrag spezifiziert ist, realisiert
  werden)

  &Aufruf einer Schnittstelle zur Lagerverwaltung \\
% }}}
\hline
\end{tabu}

\textbf{Erl\"auterung} \\

Wir legen redundant zur Lagerverwaltung unsere eingene Verwaltung
mit Angaben zum Zutatenbestand an, um auch bei Nichterreichbarkeit
der Lagerverwaltung funktionsf\"ahig zu bleiben, da unser Subsystem
essentiell f\"ur den Umsatz verantwortlich ist und ein Ausfall, das
hei{\ss}t in diesem Fall der Zustand, dass eine Zutat nicht mehr in
ben\"otigter Menge im Lager zur Verf\"ugung steht, nicht auf Grund
technischer Probleme eintreten sollte.

Allerdings stellen wir keinen Anspruch auf absolute Richtigkeit unserer
Zutantenbestandsverwaltung, da wir nur die Ereignisse unseres Subsystems,
das hei{\ss}t in diesem Fall die Entnahme einer Zutat zur Zubereitung,
protokollieren und die restlichen Angaben aus der Lagerverwaltung stammen.

Ist diese nun nicht erreichbar, verwendet unsere Zutatenbestandsverwaltung
mitunter veraltete Daten, was wir nicht mit einbeziehen.

Der Lagerverwaltung wird die Entnahme von unserem Subsystem aus mitgeteilt.

F\"ur den kompletten Synchronisationsprozess zwischen den beiden Systemen
stellt uns die Lagerverwaltung zwei Schnittstellen (oder eine, die beide
Aufgaben - Entnahme mitteilen und Zutatenbestand abfragen - zusammenfasst)
zur Verf\"ugung.

% }}}

% }}}

% Teilaufgabe 2 {{{
\subsection{Teilaufgabe 2: Ermittlung der verschiedenen Komponenten-Typen}

% }}}

% Teilaufgabe 3 {{{
\subsection{Teilaufgabe 3: Komponentendiagramm}

% }}}
