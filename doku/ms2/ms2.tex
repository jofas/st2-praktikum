\section{Meilenstein 2 $-$ Komponentenschnitt}

% Teilaufgabe 1 {{{
\subsection{Teilaufgabe 1: Vorbereitung des Komponentenschnitts}

% Liste der Geschaeftsobjekte {{{
\subsubsection{Liste der Gesch\"aftsobjekte}

\begin{itemize}

  \item Arbeitsplatz

  \item Bestellung

  \item Gericht

  \item Sitzplatz

  \item Speisekarte

  \item Zubereitungsanleitung

  \item Zutat

  \item Zutatenangabe

\end{itemize}
% }}}

% Liste der Use Cases {{{
\subsubsection{Liste der Use Cases}

\begin{itemize}

  \item Am Arbeitsplatz an-/abmelden

  \item Gericht bestellen

  \item Gericht zubereiten

\end{itemize}
% }}}

% Liste der Umsysteme {{{
\subsubsection{Liste der Umsysteme}

\begin{tabu} to \linewidth {X|X|X}
% Headerzeile {{{
\hline
\rowcolor{codebordercolor}
Umsystem &Was geschieht zwischen Umsystem und unserem Subsystem?
  &Schnittstelle angeboten oder aufgerufen \\
% }}}
% Rezeptverwaltung {{{
\hline
Rezeptverwaltung &Rezeptverwaltung verwaltet die Gesch\"aftsobjekte Gericht,
  Zubereitungsanleitung und Speisekarte. Der Gast fragt \"uber das ihm zur
  Verf\"ugung gestellte Frontend die Speisekarte und die Gerichte ab, w\"ahrend
  der Koch an seinem Terminal die Zubereitungsanleitung und die hiermit verbundenen
  Zutatenangaben, angezeigt bekommt.

  &Aufruf einer Schnittstelle zur Rezeptverwaltung \\
% }}}
% Lagerverwaltung {{{
\hline
Lagerverwaltung &Abfrage zum Zutatenbestand &Aufruf einer Schnittstelle zur
  Lagerverwaltung \\
% }}}
% Lagerverwaltung {{{
\hline
Lagerverwaltung &Angabe zur Zutantenentnahme (kann auch \"uber die gleiche
  Schnittstelle, die im obrigen Tabelleneintrag spezifiziert ist, realisiert
  werden)

  &Aufruf einer Schnittstelle zur Lagerverwaltung \\
% }}}
% Buchhaltung {{{
\hline
Buchhaltung &Abfrage der Bestellungen &Schnittstelle wird Buchhaltung zur verf\"ugung
  gestellt \\
% }}}
\hline
\end{tabu}

% Erlaeuterung {{{
\textbf{Erl\"auterung} \\

Wir legen redundant zur Lagerverwaltung unsere eingene Verwaltung
mit Angaben zum Zutatenbestand an, um auch bei Nichterreichbarkeit
der Lagerverwaltung funktionsf\"ahig zu bleiben, da unser Subsystem
essentiell f\"ur den Umsatz verantwortlich ist und ein Ausfall, das
hei{\ss}t in diesem Fall der Zustand, dass eine Zutat nicht mehr in
ben\"otigter Menge im Lager zur Verf\"ugung steht, nicht auf Grund
technischer Probleme eintreten sollte.

Allerdings stellen wir keinen Anspruch auf absolute Richtigkeit unserer
Zutantenbestandsverwaltung, da wir nur die Ereignisse unseres Subsystems,
das hei{\ss}t in diesem Fall die Entnahme einer Zutat zur Zubereitung,
protokollieren und die restlichen Angaben aus der Lagerverwaltung stammen.

Ist diese nun nicht erreichbar, verwendet unsere Zutatenbestandsverwaltung
mitunter veraltete Daten, was wir nicht mit einbeziehen.

Der Lagerverwaltung wird die Entnahme von unserem Subsystem aus mitgeteilt.

F\"ur den kompletten Synchronisationsprozess zwischen den beiden Systemen
stellt uns die Lagerverwaltung zwei Schnittstellen (oder eine, die beide
Aufgaben - Entnahme mitteilen und Zutatenbestand abfragen - zusammenfasst)
zur Verf\"ugung.

Zus\"atzlich haben wir eine Schnittstelle f\"ur die Buchhaltung angelegt.
Diese ist zwar kein explizites Subsystem, wird aber, unserer Meinung nach,
im Betriebsumfeld h\"ochstwahrscheinlich als eigenes Subsystem existieren
und unsere Schnittstelle zu den Bestellungen (im Endeffekt der Unternehmens\-
umsatz aus dem Hauptgesch\"aft) nutzen wollen.
% }}}

% }}}

% }}}

% Teilaufgabe 2 {{{
\subsection{Teilaufgabe 2: Ermittlung der verschiedenen Komponenten-Typen}

% Schritt 1 {{{
\subsubsection{Schritt 1: Gesch\"aftsobjekte in zusammenh\"angende Gruppen einteilen}

\begin{tabu} to \linewidth {X|X|X}
% Headerzeile {{{
\hline
\rowcolor{codebordercolor}
Datenkomponente &Zugeordnete Gesch\"aftsobjekte &Erkl\"arung \\
% }}}
% Bestellungskomponente {{{
\hline
Bestelldaten &Bestellung &Die einzigen Daten die in diesem Subsystem tas\"achlich generiert
  werden. Da die Bestellungen sehr wichtig f\"ur das Hauptgesch\"aft der Firma ist, es das einzige
  Datenobjekt mit Implementierung eines Create-Interfaces (Factory) ist und auch sonst nicht in unsere
  sonstigen Datenkomponenten passt, wird die Bestellung, unserer Meinung nach, in einer eigenen
  Komponente implementiert.  \\
% }}}
% Standortkomponente {{{
\hline
Standortdaten &Arbeitsplatz, Sitzplatz &Diese Daten \"andern sich \"au{\ss}erst selten (und auch nicht
  in unserem Subsystem) und umfassen im Vergleich zu anderen Komponenten wenig Datens\"atze und k\"onnen
  deshalb, unserer Meinung nach, gut zusammengefasst werden.\\
% }}}
% Gerichtskomponente {{{
\hline
Gerichtsdaten &Gericht, Speisekarte, Zubereitungsanleitung, Zutat, Zutatenangabe
  &TODO  \\
% }}}
\hline
\end{tabu}

% }}}

% Schritt 2 {{{
\subsubsection{Schritt 2: Use Cases auf Daten/Logik analysieren}

\begin{tabu} to \linewidth {X|X|X}
% Headerzeile {{{
\hline
\rowcolor{codebordercolor}
Daten-/Logikkomponente &Zugeordnete(r) Use Case(s) &Erkl\"arung \\
% }}}
% Vergabewarteschlange {{{
\hline
Vergabewarteschlange (Logik) &Am Arbeitsplatz an-/abmelden, Gericht zubereiten & TODO \\
% }}}
TODO & TODO &TODO \\
\hline
\end{tabu}

% }}}

% Schritt 3 {{{
\subsubsection{Schritt 3: Use Cases auf Nutzer-Interaktion analysieren}

\begin{tabu} to \linewidth {X|X|X|X}
% Headerzeile {{{
\hline
\rowcolor{codebordercolor}
Dialogkomponente &Zugeordnete(r) Use Case(s) &Eigene Fassadenkomponente sinnvoll?
  &Erkl\"arung \\
% }}}
% Koch-UI {{{
\hline
Koch-UI &Am Arbeitsplatz an-/abmelden, Gericht zubereiten &Ja &Fassadenkomponente zur
  Orchestrierung der Gerichtszubereitung. \\
% }}}
% Gast-UI {{{
\hline
Gast-UI &Gericht bestellen &Ja &Fassadenkomponente zur Orchestrierung des Bestellvorgangs. \\
% }}}
\hline
\end{tabu}

% }}}

% Schritt 4 {{{
\subsubsection{Schritt 4: Angebot von externen Schnittstellen}

\begin{tabu} to \linewidth {X|X|X}
% Headerzeile {{{
\hline
\rowcolor{codebordercolor}
Umsystem/Schnittstelle &Eigene Fassadenkomponente sinnvoll? &Erkl\"arung \\
% }}}
% Buchhaltung {{{
\hline
Buchhaltung &Ja &Da die Buchhaltung lesenden Zugriff auf
  usere Bestellungen haben soll, ist es notwendig eine spezialisierte Kompo\-
  nente hierf\"ur anzulegen und nicht, wie intern in unserem Subsystem, den Zugriff
  \"uber die Bestelldatenkomponente zu regeln.\\
% }}}
% Lagerverwaltung {{{
\hline
Lagerverwaltung &Nein &Zugriff erfolgt nur aus der Gerichtsdatenkomponente \"uber die
  Adapterkomponente der Lagerverwaltung, weshalb, unserer Meinung nach, keine Fassaden\-
  komponente notwendig ist. \\
% }}}
% Rezeptverwaltung {{{
\hline
Rezeptverwaltung &Nein &Zugriff erfolgt nur aus der Gerichtsdatenkomponente \"uber die
  Adapterkomponente der Rezepteverwaltung, weshalb, unserer Meinung nach, keine Fassaden\-
  komponente notwendig ist. \\
% }}}
\hline
\end{tabu}

% }}}

% Schritt 5 {{{
\subsubsection{Schritt 5: Aufruf von externen Schnittstellen/Umsystemen}

\begin{tabu} to \linewidth {X|X|X}
% Headerzeile {{{
\hline
\rowcolor{codebordercolor}
Umsystem/Schnittstelle &Adapterkomponente sinnvoll? &Erkl\"arung \\
% }}}
% Buchhaltung {{{
\hline
Buchhaltung &Nein &Bereits spezialisierte Fassadenkomponente vorhanden. \\
% }}}
% Lagerverwaltung {{{
\hline
Lagerverwaltung &Ja &Adapterkomponente f\"ur unsere Gerichtsdatenkomponente,
  die die Lese- und Schreibforg\"ange zur Verf\"ugung stellt und gleichzeitig
  bei Ausf\"allen als "`Anti-Corruption-Layer"' fungiert. \\
% }}}
% Rezeptverwaltung {{{
\hline
Rezeptverwaltung &Ja &Adapterkomponente f\"ur unsere Gerichtsdatenkomponente,
  die die Leseforg\"ange zur Verf\"ugung stellt. \\
% }}}
\hline
\end{tabu}

% }}}

% }}}

% Teilaufgabe 3 {{{
\subsection{Teilaufgabe 3: Komponentendiagramm}

% }}}
