\documentclass[12pt, twoside]{article}
\usepackage{amsmath}
\usepackage{amssymb}
\usepackage[colorlinks=true, linkcolor=black]{hyperref} % Links
\usepackage{makeidx} % Indexierung
\usepackage{siunitx}
\usepackage[ngerman]{babel} % deutsche Sonderzeichen
\usepackage[utf8]{inputenc}
\usepackage{geometry} % Dokumentendesign wie Seiten- oder Zeilenabstand bestimmen

% Graphiken
\usepackage{tikz}
\usepackage{pgfplots}
\usepackage{pgfcore}
\usepackage{pgfopts}
\usepackage{pgf}
\usepackage{ifthen}
\usepackage{booktabs}

% Tabellen
\usepackage{tabu}
\usepackage{longtable}
\usepackage{colortbl} % Tabellen faerben
\usepackage{multirow}
\usepackage{diagbox} % Tabellenzelle diagonal splitten

\usepackage{xcolor} % Farben
\usepackage[framemethod=tikz]{mdframed} % Hintergrunderstellung
\usepackage{enumitem} % Enumerate mit Buchstaben nummerierbar machen
\usepackage{pdfpages}
\usepackage{listings} % Source-Code darstellen
\usepackage{eurosym} % Eurosymbol
\usepackage[square,numbers]{natbib}
\usepackage{here} % figure an richtiger Stelle positionieren
\usepackage{verbatim} % Blockkommentare mit \begin{comment}...\end{comment}
\usepackage{ulem} % \sout{} (durchgestrichener Text)

% BibLaTex
\bibliographystyle{acm}

% mdframed Style
\mdfdefinestyle{codebox}{
	linewidth=2.5pt,
	linecolor=codebordercolor,
	backgroundcolor=codecolor,
	shadow=true,
	shadowcolor=black!40!white,
	fontcolor=black,
	everyline=true,
}

% Seitenabstaende
\geometry{left=15mm,right=15mm,top=15mm,bottom=20mm}

% TikZ Bibliotheken
\usetikzlibrary{
    arrows,
    arrows.meta,
    decorations,
    backgrounds,
    positioning,
    fit,
    petri,
    shadows,
    datavisualization.formats.functions,
    calc,
    shapes.multipart
}

% Fuer ERM "Kraehenfuss-Notation"
\tikzset{
    sep_tip/.tip = {Kite[white,length=0.1,sep=16pt]}
}
\tikzset{
    one /.tip = {|[scale=2]Square[length=8pt,width=1pt]}
}
\tikzset{
    c_one /.tip = {Circle[fill=white,scale=1.5]one}
}
\tikzset{
    n /.tip = {
        |[scale=2,sep=-2pt]
        Straight Barb[reversed,length=10pt,width=15pt,sep=-12pt]
        Square[length=12pt,width=1pt]
    }
}
\tikzset{
  cn /.tip = {Circle[fill=white,scale=1.5]n}
}
\tikzset{
    entity/.style={
        rectangle split,
        rectangle split ignore empty parts=true,
        rectangle split parts=2,label=#1,
        rounded corners,
        solid,
        draw}
}
\tikzset{
    eerm-subtype/.style={
        semicircle,
        fill=white,
        draw,
        label=\textbf{#1}
    }
}


\pgfplotsset{width=7cm,compat=1.15}

\definecolor{codecolor}{HTML}{EEEEEE}
\definecolor{codebordercolor}{HTML}{CCCCCC}

% Standardeinstellungen fuer Source-Code
\lstset{
    language=C,
    breaklines=true,
    keepspaces=true,
    keywordstyle=\bfseries\color{green!70!black},
    basicstyle=\ttfamily\color{black},
    commentstyle=\itshape\color{purple},
    identifierstyle=\color{blue},
    stringstyle=\color{orange},
    showstringspaces=false,
    rulecolor=\color{black},
    tabsize=2,
    escapeinside={\%*}{*\%},
}

\input{libuml}
%\input{liberm}

\begin{document}

% Custom Titelseite
\begin{titlepage}
	\begin{center}
		{\Huge{\underline{\textbf{Softwaretechnik II $-$ Praktikum}}}} \\
	\vspace{3cm}
		{\Huge{\textbf{Subsystem 4 $-$ Zubereitung}}} \\
	\vspace{3cm}
		{\Huge{Eine Dokumentation von:}} \\
	\vspace{2cm}
		\huge{J. Fa{\ss}bender} \\
	\vspace{.5cm}
		\huge{J. Gobelet} \\
	\vspace{.5cm}
		\huge{L. Gobelet} \\
	\vspace{.5cm}
		\huge{E. G\"odel}
	\end{center}
\end{titlepage}

\tableofcontents
\newpage
\listoffigures

\section{Meilenstein 1 $-$ Datenzugriffsschicht}

\subsection{Teilaufgabe 1: Ausschnitt aus Logischem DM mit Entities und Value Objects}

\subsubsection{Klassendiagramm}
\begin{center}
\begin{tikzpicture}

% Ebene 0 {{{
  \UMLClassAlterName
    {0}
    {0}
    {
      \textbf{Gericht}
        \nodepart{second}
      Name: String \\
      Details: String \\
      Preis: double
    }
    {gericht}
    {minimum width=6cm,text width=2.75cm}

  \UMLClassRelativeToAlterName
    {right=3 of gericht}
    {\textbf{Zutat}\nodepart{second}Name: String}
    {zutat}
    {minimum width=6cm}
% }}}

% Ebene -1 {{{
  \UMLClassRelativeToAlterName
    {below=3 of gericht}
    {\textbf{Speise}\nodepart{second}Name: String}
    {speise}
    {minimum width=6cm}
% }}}

% Ebene -2 {{{
  \UMLClassRelativeToAlterName
    {below=3 of speise}
    {
      \textbf{Zubereitungsanleitung}
        \nodepart{second}
      Anleitung: String
    }
    {za}
    {minimum width=6cm,fill=red!25}

  \UMLClassRelativeToAlterName
    {right=3 of za}
    {\textbf{Zutatenangabe}\nodepart{second}Menge: int}
    {zutatenangabe}
    {minimum width=6cm,fill=red!25}
% }}}

% Verbindungen {{{
  % gericht -- speise {{{
  \draw[umlaggreg-] (gericht.south) -- (speise.north);
  \node[below left=.25 of gericht.south] {*};
  \node[above left=.25 of speise.north] {*};
  \node[left] at ($(gericht)!.5!(speise)$) {Teil von};
  % }}}
  % speise -- za {{{
  \draw[umlcompo-] (speise.south) -- (za.north);
  \node[below left=.25 of speise.south] {1};
  \node[above left=.25 of za.north] {1};
  \node[left] at ($(speise)!.5!(za)$) {beschreibt};
  % }}}
  % zutatenangabe -- za {{{
  \draw[-umlcompo] (zutatenangabe.west) -- (za.east);
  \node[above left=.25 of zutatenangabe.west] {1..*};
  \node[above right=.25 of za.east] {1};
  \node[below] at ($(zutatenangabe)!.5!(za)$) {ben\"otigt};
  % }}}
  % zutatenangabe -- zutat {{{
  \draw (zutatenangabe.north) -- (zutat.south);
  \node[above left=.25 of zutatenangabe.north] {*};
  \node[below left=.25 of zutat.south] {1};
  \node[left] at ($(zutatenangabe)!.5!(zutat)$) {ben\"otigt};
  % }}}
% }}}
\end{tikzpicture}
\end{center}


\subsubsection{Fachliches Glossar}
\begin{tabu} to \linewidth {X|X|X}
% Headerzeile {{{
\hline
\rowcolor{codebordercolor}
Gesch\"aftsobjekt &Attribut &Erkl\"arung \\
% }}}
% Gericht {{{
\hline
Gericht & &Vom Restaurant angebotenes Mahl. \\
  % Name {{{
  \hline
  &Name &Gerichtsbezeichnung. \\
  % }}}
  % Details {{{
  \hline
  &Details &Wird dem Gast angezeigt. Enth\"alt n\"ahere Angaben zu den Zutaten. \\
  % }}}
  % Preis {{{
  \hline
  &Preis &Geldbetrag der f\"ur das Gericht zu bezahlen ist. \\
  % }}}
% }}}
% Speise {{{
\hline
Speise & &Teil eines Gerichts. Beispielsweise w\"are eine Salatbeilage als Speise zu verstehen. \\
  % Name {{{
  \hline
  &Name &Bezeichnung der Speise. \\
  % }}}
% }}}
% Zubereitungsanleitung {{{
\hline
Zubereitungsanleitung & &Leitfaden zur Zubereitung einer Speise. \\
  % Anleitung {{{
  \hline
  &Anleitung &Erkl\"arender Text, der beschreibt, wie eine Speise zuzubereiten ist. \\
  % }}}
% }}}
% Zutat {{{
\hline
Zutat & &Ben\"otigt f\"ur die Zubereitung einer Speise. \\
  % Name {{{
  \hline
  &Name &Bezeichnung der Zutat. \\
  % }}}
% }}}
% Zutatenangabe {{{
\hline
Zutatenangabe & &Zuordnung zwischen Zutat und Zubereitungsanleitung.
                 Gibt die Menge einer Zutat an, die f\"ur die Zubereitung notwendig ist. \\
  % Menge {{{
  \hline
  &Menge &Die ben\"otigte Menge. \\
  % }}}
% }}}
\hline
\end{tabu}


\subsubsection{Erweiterungen der Aufgabenstellung}

Da es in unserem Logischen Datenmodell keine 1:1-Beziehung gab,
haben wir eine zus\"atzliche redundante Entit\"at eingebaut.

Hierbei handelt es sich um die Entit\"at Speise. Diese Entit\"at
h\"atte genauso gut einfach Teil der Zubereitungsanleitung sein
k\"onnen und ist nur in unser Modell aufgenommen worden, damit
wir die f\"ur die Aufgabenstellung ben\"otigte 1:1-Beziehung in
unserem Diagramm haben.


\end{document}
